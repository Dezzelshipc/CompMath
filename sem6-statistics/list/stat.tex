\documentclass[14pt, a1paper, fleqn]{extarticle}

\usepackage{style/style}
% \usepackage{style/titlepage}

\everymath{\displaystyle}

\hypersetup{
    colorlinks=true,
    urlcolor=blue
}
\renewcommand{\arraystretch}{1.5}
\setlength{\tabcolsep}{4pt}


\DeclareMathOperator{\D}{\partial}
\DeclareMathOperator{\Ln}{Ln}
\DeclareMathOperator{\Imz}{Im}
\DeclareMathOperator{\Rez}{Re}
\DeclareMathOperator{\sign}{sign}



\begin{document}
    \begin{center}
        \Huge\textbf{Сводная таблица по математической статистике}
    \end{center}

    \normalsize \vspace*{-1em} Python для всех пунктов: \vspace*{-0.5em}
    \begin{enumerate}
        \item Двусторонний тест: \( \text{\hyperref[pv]{p-value}} = 2 \cdot \min\left\{ \text{r.cdf}(v_p), 1 - \text{r.cdf}(v_p) \right\} \), где r -- распределение статистики, \(v_p\) -- значение расчётной статистики. \label{pv}
        \item \( \overline{X} = \text{np.mean}(x) \)
        \item \( S_0 = \text{np.std}(x, ddof=1), ~ S_0^2 = \text{np.var}(x, ddof=1) \)
    \end{enumerate}
    \vspace*{-2em}

    \section{Одно распределение}
    \begin{center}
        \begin{tabular}{|p{6cm}|p{8cm}|p{3cm}|p{3cm}|p{9cm}|p{10cm}|p{14cm}|}
            \hline
            Название & Предпосылки & \( H_0 \) & \( H_1 \) & Статистика & Выводы & Python (numpy, scipy.stats) \\
            \hline
            Гипотеза о матожидании 
            & \begin{enumerate}
             \item \( X \sim N(\mu, \sigma^2) \)
             \item \( \sigma^2 \) - известно 
            \end{enumerate} 
            & \( \mu = \mu_0 \) 
            & \( \mu \neq \mu_0 \) 
            & \( z_p = \frac{\overline{X} - \mu_0}{\sigma / \sqrt{n}} \sim N(0, 1) \) 
            & Не отвергаем на уровне значимости \( \alpha \), если 
            \begin{enumerate}
                \item \( z_p \in \left( -z_{1-\frac{\alpha}{2}}, z_{1-\frac{\alpha}{2}} \right) \),
                \item \( \mu_0 \in \left( \overline{X}-z_{1-\frac{\alpha}{2}}\frac{\sigma}{\sqrt{n}}, \overline{X}+z_{1-\frac{\alpha}{2}}\frac{\sigma}{\sqrt{n}} \right) \)
                \item \( \text{p-value} > \alpha \)
            \end{enumerate} 
            & \begin{enumerate}
                \item \( z_{1-\frac{\alpha}{2}} = \text{norm.ppf}(q=1 - \alpha/2) \),
                \item \( \text{p-value} = 2 \cdot \left( 1 -  \text{norm.cdf}(\text{abs}(z_p)) \right) \)
            \end{enumerate} \\
            \hline
            Гипотеза о матожидании 
            & \begin{enumerate}
             \item \( X \sim N(\mu, \sigma^2) \)
             \item \( \sigma^2 \) - неизвестно 
            \end{enumerate} 
            & \( \mu = \mu_0 \) 
            & \( \mu \neq \mu_0 \) 
            & \( t_p = t^{(n-1)} = \frac{\overline{X} - \mu_0}{ S_0 / \sqrt{n}} \sim T_{n-1} \) 
            & Не отвергаем на уровне значимости \( \alpha \), если 
            \begin{enumerate}
                \item \( t_p \in \left( -t^{(n-1)}_{1-\frac{\alpha}{2}}, t^{(n-1)}_{1-\frac{\alpha}{2}} \right) \),
                \item \( \mu_0 \in \left( \overline{X}-t^{(n-1)}_{1-\frac{\alpha}{2}}\frac{S_0}{\sqrt{n}}, \overline{X}+t^{(n-1)}_{1-\frac{\alpha}{2}}\frac{S_0}{\sqrt{n}} \right) \)
                \item \( \text{p-value} > \alpha \)
            \end{enumerate} 
            & \begin{enumerate}
                \item \( t^{(n-1)}_{1-\frac{\alpha}{2}} = \text{t.ppf}(df=n-1, q=1 - \alpha/2) \),
                \item \( \text{p-value} = 2 \cdot \left( 1 - \text{t.cdf}(\text{abs}(t_p), df=n-1) \right) \)
            \end{enumerate} \\
            \hline
            Гипотеза о дисперсии 
            & \begin{enumerate}
             \item \( X \sim N(\mu, \sigma^2) \)
             \item \( \mu \) - неизвестно 
            \end{enumerate} 
            & \( \sigma^2 = \sigma^2_0 \) 
            & \( \sigma^2 \neq \sigma^2_0 \) 
            & \( C_p = C^{(n-1)} = \frac{S_0^2 (n-1)}{ \sigma_0^2 } \sim \chi^2_{n-1} \) 
            & Не отвергаем на уровне значимости \( \alpha \), если 
            \begin{enumerate}
                \item \( C_p \in \left( C^{(n-1)}_{\frac{\alpha}{2}}, C^{(n-1)}_{1-\frac{\alpha}{2}} \right) \),
                \item \( \sigma_0^2 \in \left( \frac{(n-1) S_0^2}{C^{(n-1)}_{1-\frac{\alpha}{2}}}, \frac{(n-1) S_0^2}{C^{(n-1)}_{\frac{\alpha}{2}}} \right) \)
                \item \( \text{p-value} > \alpha \)
            \end{enumerate} 
            & \begin{enumerate}
                \item \( C^{(n-1)}_{\frac{\alpha}{2}} = \text{chi2.ppf}(df=n-1, q=\alpha/2) \),
                \item \( C^{(n-1)}_{1-\frac{\alpha}{2}} = \text{chi2.ppf}(df=n-1, q=1 - \alpha/2) \),
                \item \hyperref[pv]{p-value}
            \end{enumerate} \\
            \hline
            Гипотеза о дисперсии 
            & \begin{enumerate}
             \item \( X \sim N(\mu, \sigma^2) \)
             \item \( \mu \) - известно 
            \end{enumerate} 
            & \( \sigma^2 = \sigma^2_0 \) 
            & \( \sigma^2 \neq \sigma^2_0 \) 
            & \( C_p = C^{(n)} = \frac{\sum_{i=1}^{n} \left( x_i - \mu \right)^2 }{ \sigma_0^2 } \sim \chi^2_{n} \) 
            & Не отвергаем на уровне значимости \( \alpha \), если 
            \begin{enumerate}
                \item \( C_p \in \left( C^{(n)}_{\frac{\alpha}{2}}, C^{(n)}_{1-\frac{\alpha}{2}} \right) \),
                \item \( \sigma_0^2 \in \left( \frac{\sum_{i=1}^{n} \left( x_i - \mu \right)^2}{C^{(n-1)}_{1-\frac{\alpha}{2}}}, \frac{\sum_{i=1}^{n} \left( x_i - \mu \right)^2}{C^{(n-1)}_{\frac{\alpha}{2}}} \right) \)
                \item \( \text{p-value} > \alpha \)
            \end{enumerate} 
            & \begin{enumerate}
                \item \( C^{(n-1)}_{\frac{\alpha}{2}} = \text{chi2.ppf}(df=n-1, q=\alpha/2) \),
                \item \( C^{(n-1)}_{1-\frac{\alpha}{2}} = \text{chi2.ppf}(df=n-1, q=1 - \alpha/2) \),
                \item \hyperref[pv]{p-value}
            \end{enumerate} \\
            \hline
            Асимптотическая гипотеза о матожидании 
            & \begin{enumerate}
             \item \( X \sim \mathcal{F} \)
             \item \( D(x) = \sigma^2 \) - известно 
             \item \( n \to \infty ~ (n \gg 0) \)
            \end{enumerate} 
            & \( \mu = \mu_0 \) 
            & \( \mu \neq \mu_0 \) 
            & \( z_p = \frac{\overline{X} - \mu_0}{\sigma / \sqrt{n}} \xrightarrow[n \to \infty]{d} N(0, 1) \) 
            & Не отвергаем на уровне значимости \( \alpha \), если 
            \begin{enumerate}
                \item \( z_p \in \left( -z_{1-\frac{\alpha}{2}}, z_{1-\frac{\alpha}{2}} \right) \),
                \item \( \mu_0 \in \left( \overline{X}-z_{1-\frac{\alpha}{2}}\frac{\sigma}{\sqrt{n}}, \overline{X}+z_{1-\frac{\alpha}{2}}\frac{\sigma}{\sqrt{n}} \right) \)
                \item \( \text{p-value} > \alpha \)
            \end{enumerate} 
            & \begin{enumerate}
                \item \( z_{1-\frac{\alpha}{2}} = \text{norm.ppf}(q=1 - \alpha/2) \),
                \item \( \text{p-value} = 2 \cdot \left( 1 -  \text{norm.cdf}(\text{abs}(z_p)) \right) \)
            \end{enumerate} \\
            \hline
            Асимптотическая гипотеза о матожидании 
            & \begin{enumerate}
             \item \( X \sim \mathcal{F} \)
             \item \( D(x) = \sigma^2 \) - неизвестно 
             \item \( n \to \infty ~ (n \gg 0) \)
            \end{enumerate} 
            & \( \mu = \mu_0 \) 
            & \( \mu \neq \mu_0 \) 
            & \( z_p = \frac{\overline{X} - \mu_0}{S_0 / \sqrt{n}} \xrightarrow[n \to \infty]{d} N(0, 1) \) 
            & Не отвергаем на уровне значимости \( \alpha \), если 
            \begin{enumerate}
                \item \( z_p \in \left( -z_{1-\frac{\alpha}{2}}, z_{1-\frac{\alpha}{2}} \right) \),
                \item \( \mu_0 \in \left( \overline{X}-z_{1-\frac{\alpha}{2}}\frac{S_0}{\sqrt{n}}, \overline{X}+z_{1-\frac{\alpha}{2}}\frac{S_0}{\sqrt{n}} \right) \)
                \item \( \text{p-value} > \alpha \)
            \end{enumerate} 
            & \begin{enumerate}
                \item \( z_{1-\frac{\alpha}{2}} = \text{norm.ppf}(q=1 - \alpha/2) \),
                \item \( \text{p-value} = 2 \cdot \left( 1 -  \text{norm.cdf}(\text{abs}(z_p)) \right) \)
            \end{enumerate} \\
            \hline
            Bootstrap
            & \begin{enumerate}
             \item \( X \sim \mathcal{F} \)
             \item \( n \) - небольшое
            \end{enumerate} 
            & \(  \mu = \mu_0 \) \newline 
            или \newline
             \( \sigma^2 = \sigma^2_0 \)
            & \(  \mu \neq \mu_0 \) \newline или \newline \( \sigma^2 \neq \sigma^2_0 \)
            & Генерируем много выборок из данной одинаковой длины. Считаем для каждой них нужную статистику \( \left( \overline{X_i} ~ \text{или} ~ \widehat{\sigma}_i^2 \right) \). Считаем квантили \( q_{\frac{\alpha}{2}}, q_{1-\frac{\alpha}{2}} \) для выборки этих статистик.
            & Не отвергаем на уровне значимости \( \alpha \), если 
            \begin{enumerate}
                \item \( 
                    \mu_0 ~ (\sigma^2_0) \in \left( q_{\frac{\alpha}{2}}, q_{1-\frac{\alpha}{2}} \right) \)
            \end{enumerate} 
            & \begin{enumerate}
                \item \href{https://docs.scipy.org/doc/scipy/reference/generated/scipy.stats.bootstrap.html}{scipy.stats.bootstrap}
                \item \href{https://numpy.org/doc/stable/reference/random/generated/numpy.random.choice.html}{numpy.random.choice}
            \end{enumerate} \\
            \hline
        \end{tabular}
    \end{center}


    \section{Два распределения}
    \begin{center}
        \begin{tabular}{|p{6cm}|p{8cm}|p{3cm}|p{3cm}|p{9cm}|p{10cm}|p{14cm}|}
            \hline
            Название & Предпосылки & \( H_0 \) & \( H_1 \) & Статистика & Выводы & Python (numpy, scipy.stats) \\
            \hline
            Гипотеза о разности матожиданий связанных пар 
            & \begin{enumerate}
             \item \( X \sim N(\mu_x, \sigma^2_x) \)
             \item \( Y \sim N(\mu_y, \sigma^2_y) \)
             \item \( n = n_x = n_y \)
             \item \( \sigma^2_x, \sigma^2_y \) -- известно 
            \end{enumerate} 
            & \( \mu_x - \mu_y = \mu_0 \) 
            & \( \mu_x - \mu_y \neq \mu_0 \) 
            & \( \Delta = X - Y \), \newline 
            \( z_p = \frac{\overline{\Delta} - \mu_0}{D( \overline{\Delta})} \sim N(0, 1) \), \newline 
            \( z_p = \frac{\overline{X} - \overline{Y} - (\mu_x - \mu_y)}{ \sqrt{\frac{\sigma_x^2 + \sigma_y^2}{n}} } \sim N(0, 1) \) 
            & Не отвергаем на уровне значимости \( \alpha \), если 
            \begin{enumerate}
                \item \( z_p \in \left( -z_{1-\frac{\alpha}{2}}, z_{1-\frac{\alpha}{2}} \right) \),
                \item \( \mu_0 \in \left( \overline{\Delta}-z_{1-\frac{\alpha}{2}} D( \overline{\Delta}), \overline{\Delta}+z_{1-\frac{\alpha}{2}} D( \overline{\Delta}) \right) \)
                \item \( \text{p-value} > \alpha \)
            \end{enumerate} 
            & См. аналогичное выше, \hyperref[pv]{p-value} \\
            \hline
            Гипотеза о разности матожиданий связанных пар 
            & \begin{enumerate}
             \item \( X \sim N(\mu_x, \sigma^2_x) \)
             \item \( Y \sim N(\mu_y, \sigma^2_y) \)
             \item \( n = n_x = n_y \)
             \item \( \sigma^2_x, \sigma^2_y \) -- неизвестно 
            \end{enumerate} 
            & \( \mu_x - \mu_y = \mu_0 \) 
            & \( \mu_x - \mu_y \neq \mu_0 \) 
            & \( \Delta = X - Y \), \newline 
            \( t_p = \frac{\overline{\Delta} - \mu_0}{S_0(\Delta) / \sqrt{n}} \sim T_{n-1} \)
            & Не отвергаем на уровне значимости \( \alpha \), если 
            \begin{enumerate}
                \item \( t_p \in \left( -t^{(n-1)}_{1-\frac{\alpha}{2}}, t^{(n-1)}_{1-\frac{\alpha}{2}} \right) \),
                \item \( \mu_0 \in \left( \overline{\Delta}\pm t^{(n-1)}_{1-\frac{\alpha}{2}} \frac{S_0(\Delta)}{ \sqrt{n}} \right) \)
                \item \( \text{p-value} > \alpha \)
            \end{enumerate} 
            & См. аналогичное выше, \hyperref[pv]{p-value} \\
            \hline
            Гипотеза о разности матожиданий
            & \begin{enumerate}
             \item \( X \sim N(\mu_x, \sigma^2_x) \)
             \item \( Y \sim N(\mu_y, \sigma^2_y) \)
             \item \( \sigma^2_x, \sigma^2_y \) -- известно 
            \end{enumerate} 
            & \( \mu_x - \mu_y = \mu_0 \) 
            & \( \mu_x - \mu_y \neq \mu_0 \) 
            & \( z_p = \frac{\overline{X} - \overline{Y} - (\mu_x - \mu_y)}{ \sqrt{\frac{\sigma_x^2}{n_x} + \frac{\sigma_y^2}{n_y}} } \sim N(0, 1) \) 
            & Не отвергаем на уровне значимости \( \alpha \), если 
            \begin{enumerate}
                \item \( z_p \in \left( -z_{1-\frac{\alpha}{2}}, z_{1-\frac{\alpha}{2}} \right) \),
                \item \( (\mu_x - \mu_y) \in \left( \overline{X} - \overline{Y} \pm z_{1-\frac{\alpha}{2}} \sqrt{\frac{\sigma_x^2}{n_x} + \frac{\sigma_y^2}{n_y}} \right) \)
                \item \( \text{p-value} > \alpha \)
            \end{enumerate} 
            & См. аналогичное выше, \hyperref[pv]{p-value} \\
            \hline
        \end{tabular}
    \end{center}

    
    \begin{center}
        \begin{tabular}{|p{6cm}|p{8cm}|p{3cm}|p{3cm}|p{9cm}|p{10cm}|p{14cm}|}
            \hline
            Название & Предпосылки & \( H_0 \) & \( H_1 \) & Статистика & Выводы & Python (numpy, scipy.stats) \\
            \hline
            Гипотеза о разности матожиданий
            & \begin{enumerate}
             \item \( X \sim N(\mu_x, \sigma^2) \)
             \item \( Y \sim N(\mu_y, \sigma^2) \)
             \item \( \sigma^2 = \sigma^2_x = \sigma^2_y \) -- неизвестно 
            \end{enumerate} 
            & \( \mu_x - \mu_y = \mu_0 \) 
            & \( \mu_x - \mu_y \neq \mu_0 \) 
            & \( \widehat{\sigma}^2 = \frac{S_0^2 (X) (n_x - 1) + S_0^2 (Y) (n_y - 1)}{n_x + n_y - 2} \)
            \newline
            \( t_p = \frac{\overline{X} - \overline{Y} - (\mu_x - \mu_y)}{ \widehat{\sigma} \sqrt{\frac{1}{n_x} + \frac{1}{n_y}} } \sim T_{n_x+n_y-2} \) 
            & Не отвергаем на уровне значимости \( \alpha \), если 
            \begin{enumerate}
                \item \( t_p \in \left( -t^{(n_x+n_y-2)}_{1-\frac{\alpha}{2}}, t^{(n_x+n_y-2)}_{1-\frac{\alpha}{2}} \right) \),
                \item \( (\mu_x - \mu_y) \in \left( \overline{X} - \overline{Y} \pm t^{(n_x+n_y-2)}_{1-\frac{\alpha}{2}} \hat{ \sigma } \right) \)
                \item \( \text{p-value} > \alpha \)
            \end{enumerate} 
            & \begin{enumerate}
                \item \( t_{1-\frac{\alpha}{2}}^{(n_x+n_y-2)} = \text{t.ppf}(q=1 - \alpha/2 ,df = n_x+n_y-2) \),
                \item \( \text{p-value} = 2 \cdot \left( 1 - \text{t.cdf}(\text{abs}(t_p), df=n_x+n_y-2) \right) \),
                \item \href{https://docs.scipy.org/doc/scipy/reference/generated/scipy.stats.ttest_ind.html}{scipy.stats.ttest\_ind}
            \end{enumerate} \\
            \hline
            Гипотеза о \textit{равенстве} матожиданий. Тест Уэлча
            & \begin{enumerate}
             \item \( X \sim N(\mu_x, \sigma_x^2) \)
             \item \( Y \sim N(\mu_y, \sigma_y^2) \)
             \item \( \sigma^2_x, \sigma^2_y \) -- неизвестно 
            \end{enumerate} 
            & \( \mu_x - \mu_y = 0 \) 
            & \( \mu_x - \mu_y \neq 0 \) 
            & \( t_p = \frac{\overline{X} - \overline{Y}}{ \sqrt{ \dfrac{\widehat{\sigma_x^2}}{n_x} + \dfrac{\widehat{\sigma_y^2}}{n_y} } } \sim T_{\widehat{d}} \) 
            \newline
            \( \widehat{d} = \frac{ \left( \dfrac{\widehat{\sigma_x^2}}{n_x} + \dfrac{\widehat{\sigma_y^2}}{n_y}\right)^2 }{\dfrac{\widehat{\sigma}_x^4}{n^2_x(n_x-1)} + \dfrac{\widehat{\sigma}_y^4}{n^2_y(n_y-1)}} \)
            & Не отвергаем на уровне значимости \( \alpha \), если 
            \begin{enumerate}
                \item \( t_p \in \left( -t^{(\widehat{d})}_{1-\frac{\alpha}{2}}, t^{(\widehat{d})}_{1-\frac{\alpha}{2}} \right) \),
                \item \( 0 \in \left( \overline{X} - \overline{Y} \pm t^{(\widehat{d})}_{1-\frac{\alpha}{2}} \sqrt{ \dfrac{\widehat{\sigma_x^2}}{n_x} + \dfrac{\widehat{\sigma_y^2}}{n_y} } \right) \)
                \item \( \text{p-value} > \alpha \)
            \end{enumerate} 
            & \begin{enumerate}
                \item \( t_{1-\frac{\alpha}{2}} = \text{t.ppf}(q=1 - \alpha/2 ,df = d) \),
                \item \( \text{p-value} = 2 \cdot \left( 1 - \text{t.cdf}(\text{abs}(t_p), df=d) \right) \)
                \item \href{https://docs.scipy.org/doc/scipy/reference/generated/scipy.stats.ttest_ind.html}{scipy.stats.ttest\_ind(equal\_var=Flase)}
            \end{enumerate} \\

            \hline
            Гипотеза об отношении дисперсий
            & \begin{enumerate}
             \item \( X \sim N(\mu_x, \sigma_x^2) \)
             \item \( Y \sim N(\mu_y, \sigma_y^2) \)
             \item \( \sigma^2_x, \sigma^2_y \) -- неизвестно 
            \end{enumerate} 
            & \( \frac{\sigma_x^2}{\sigma_y^2} = 1 \) 
            & \( \frac{\sigma_x^2}{\sigma_y^2} \neq 1 \) 
            & \( f_p = \frac{\widehat{\sigma^2_x}}{\widehat{\sigma^2_y}} \sim F_{n_x-1, n_y-1} \)
            & Не отвергаем на уровне значимости \( \alpha \), если 
            \begin{enumerate}
                \item \( f_p \in \left( f^{(n_x-1, n_y-1)}_{\frac{\alpha}{2}}, f^{(n_x-1, n_y-1)}_{1-\frac{\alpha}{2}} \right) \),
                \item \( \text{p-value} > \alpha \)
            \end{enumerate} 
            & \begin{enumerate}
                \item \( f^{(n_x-1, n_y-1)}_{\frac{\alpha}{2}} = \text{f.ppf}(dfn=n_x-1, dfd = n_y-1, q=a/2) \),
                \item \( f^{(n_x-1, n_y-1)}_{1-\frac{\alpha}{2}} = \text{f.ppf}(dfn=n_x-1, dfd = n_y-1, q=1-a/2) \),
                \item \hyperref[pv]{p-value}
            \end{enumerate} \\
            \hline
        \end{tabular}
    \end{center}

    \section{Критерии сравнения}
    \begin{center}
        \begin{tabular}{|p{6cm}|p{8cm}|p{3cm}|p{3cm}|p{9cm}|p{10cm}|p{14cm}|}
            \hline
            Название & Предпосылки & \( H_0 \) & \( H_1 \) & Статистика & Выводы & Python (numpy, scipy.stats) \\
            \hline
            Критерий Пирсона (\( \chi^2 \)) о согласии
            & \begin{enumerate}
             \item \( X \sim \mathcal{F}_x \)
             \item \( \mathcal{F}_0 \) -- дискретное.
            \end{enumerate} 
            & \( \mathcal{F}_x = \mathcal{F}_0 \) 
            & \( \mathcal{F}_x \neq \mathcal{F}_0 \) 
            & Для каждого значения \( a_i \) имеем частоту/количество (\( \nu_i \)) в данной выборке  и теоретическую вероятность \( p_i \). 
            \newline
            \( \rho = \sum_{i=1}^{k} \frac{ \left( \nu_i - n p_i \right)^2 }{n p_i} \xrightarrow[n \to \infty]{H_0} \chi^2_{k-1}\)
            & Не отвергаем на уровне значимости \( \alpha \), если 
            \begin{enumerate}
                \item \( \text{p-value} > \alpha \)
            \end{enumerate} 
            & \begin{enumerate}
                \item \( \text{p-value} = 2 \cdot \text{chi2.cdf}(\rho, df=n-1) \)
            \end{enumerate} \\
            \hline
            Критерий Колмогорова о согласии
            & \begin{enumerate}
             \item \( X \sim \mathcal{F}_x \)
             \item \( \mathcal{F}_0 \) -- непрерывное.
            \end{enumerate} 
            & \( \mathcal{F}_x = \mathcal{F}_0 \) 
            & \( \mathcal{F}_x \neq \mathcal{F}_0 \) 
            & \( \widehat{F}_n (x) \) -- эмпирическая функция распределения, \( F_0(x) \) -- функция распределения \( \mathcal{F}_0 \).
            \newline
            \( D_n = \sup_x \left| \widehat{F}_n (x) - F_0 (x) \right| \),
            \newline
            \( k_p = \sqrt{n} D_n \xrightarrow[n \to \infty]{d} \eta \sim \mathcal{K}(y) \) -- функция распределения Колмогорова.
            & Не отвергаем на уровне значимости \( \alpha \), если 
            \begin{enumerate}
                \item \( k_p \leq K_{1-\alpha} \),
                \item \( \text{p-value} > \alpha \)
            \end{enumerate} 
            & \begin{enumerate}
                \item \href{https://docs.scipy.org/doc/scipy/reference/generated/scipy.stats.ksone.html}{scipy.stats.ksone}
                \item \href{https://docs.scipy.org/doc/scipy/reference/generated/scipy.stats.ks_1samp.html}{scipy.stats.ks\_1samp}
                \item \href{https://docs.scipy.org/doc/scipy/reference/generated/scipy.stats.kstest.html}{scipy.stats.kstest}
            \end{enumerate} \\
            \hline
            Критерий Колмогорова-Смирнова об однородности
            & \begin{enumerate}
             \item \( X \sim \mathcal{F}_x \)
             \item \( Y \sim \mathcal{F}_y \)
            \end{enumerate} 
            & \( \mathcal{F}_x = \mathcal{F}_y \) 
            & \( \mathcal{F}_x \neq \mathcal{F}_y \) 
            & \( \widehat{F}_{n_x} (x), \widehat{F}_{n_y} (x) \) -- эмпирические функции распределения.
            \newline
            \( ks_p = \sqrt{\frac{n_x n_y}{n_x + n_y}} \sup_x \left| \widehat{F}_{n_x} (x) - \widehat{F}_{n_y} (x) \right|\) 
            \( ks_p \xrightarrow[n_x, n_y \to \infty]{d} \eta \sim \mathcal{K}(y) \) -- функция распределения Колмогорова.
            & Не отвергаем на уровне значимости \( \alpha \), если 
            \begin{enumerate}
                \item \( ks_p \leq K_{1-\alpha} \),
                \item \( \text{p-value} > \alpha \)
            \end{enumerate} 
            & \begin{enumerate}
                \item \href{https://docs.scipy.org/doc/scipy/reference/generated/scipy.stats.ksone.html}{scipy.stats.ksone}
                \item \href{https://docs.scipy.org/doc/scipy/reference/generated/scipy.stats.ks_2samp.html}{scipy.stats.ks\_2samp}
            \end{enumerate} \\
            \hline
            Критерий Пирсона (\( \chi^2 \)) о независимости
            & Объекты имеют пары из категорий \( (x_i, y_i) \). Всего \( X \) имеет \( s \) категорий, \( Y \) имеет \( k \) категорий.
            & \( X, Y \) - независимые
            & \( X, Y \) - зависимые 
            & \( \nu_{ij} \) - частоты пары категорий \( (a_{i}, b_{j}) \sim (X,  Y) \).
            \newline
            \( n_{i \bullet} = \sum_{j=1}^{k} \nu_{ij}, ~ n_{\bullet j} = \sum_{i=1}^{s} \nu_{ij} \)
            \newline
            \( \gamma = \sum_{i=1}^{s} \sum_{j=1}^{k} \frac{\left( \nu_{ij} - \dfrac{n_{i \bullet} n_{\bullet j}}{n} \right)^2}{\dfrac{n_{i \bullet} n_{\bullet j}}{n}} \sim \chi^2_{(s-1)(k-1)} \)

            & Не отвергаем на уровне значимости \( \alpha \), если 
            \begin{enumerate}
                \item \( \gamma \in (0,  C^{(s-1)(k-1)}_{1-\alpha}) \),
                \item \( \text{p-value} > \alpha \)
            \end{enumerate} 
            & \begin{enumerate}
                \item \href{https://docs.scipy.org/doc/scipy/reference/generated/scipy.stats.contingency.crosstab.html}{scipy.stats.contingency.crosstab}
                \item \href{https://pandas.pydata.org/docs/reference/api/pandas.crosstab.html}{pandas.crosstab}
                \item \href{https://docs.scipy.org/doc/scipy/reference/generated/scipy.stats.chi2_contingency.html}{scipy.stats.chi2\_contingency} (correction = False)
            \end{enumerate} \\
            \hline
            Коэффициент корреляции Спирмена
            & Объекты имеют пары из порядковых (ранговых) переменных \( (r_i, k_i) \).
            & \( X, Y \) - независимые
            & \( X, Y \) - зависимые 
            & \( S = \sum_{i=1}^{n} (r_i - k_i)^2 \in \left[ 0, \frac{n^3 - n}{3} \right]\)
            \newline
            \( \rho = 1 - \frac{6S}{n^3-n} \in [-1, 1] \)
            \newline
            \( \rho_p = \sqrt{n-1} \rho \xrightarrow[H_0]{n \to \infty} N(0, 1) \)
            & Не отвергаем на уровне значимости \( \alpha \), если 
            \begin{enumerate}
                \item \( \rho_p \in \left( -z_{1-\frac{\alpha}{2}}, z_{1-\frac{\alpha}{2}} \right) \),
                \item \( \text{p-value} > \alpha \)
            \end{enumerate} 
            & \begin{enumerate}
                \item \href{https://docs.scipy.org/doc/scipy/reference/generated/scipy.stats.spearmanr.html}{scipy.stats.spearmanr}
                \item \hyperref[pv]{p-value}
            \end{enumerate} \\
            \hline
        \end{tabular}
    \end{center}

    Красный текст -- ссылка в этом документе. Синий текст -- ссылка на страницу в интернете.

\end{document}