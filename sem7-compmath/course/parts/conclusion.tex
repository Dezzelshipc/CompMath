\section{Заключение}

В ходе данной работы была исследована сплайн-разностная схема метода сплайн-коллокации для приближённого решения обыкновенного дифференциального уравнения 2 порядка с краевыми условиями.

Данный метод, как и аналогичные методы, использующие сплайны, хороши тем, что численное решение можно вычислить на всём отрезке задачи, в том числе и его первую производную с высокой точностью. Недостатком же данного метода является то, что его может быть достаточно непросто распространять на уравнения, у которых есть слагаемые с производной 1 порядка, из-за громоздкости получаемых выражений. Поэтому данная схема подходит только для определённых дифференциальных уравнений 2 порядка, в которых нет 1 производной.
 
В результате работы над курсовым проектом приобрел практические навыки владения:
\begin{itemize}
    \item современными численными методами решения задач математической физики;
    \item основами алгоритмизации для численного решения задач математической физики на одном из языков программирования;
    \item инструментальными средствами, поддерживающими разработку программного обеспечения для численного решения задач математической физики;
\end{itemize}
а также навыками представления итогов проделанной работы в виде отчета, оформленного в соответствии с имеющимися требованиями, с привлечением современных средств редактирования и печати.
