\section{Введение}
Объектом исследования являются численные методы решения задач математической физики, а также программное обеспечение, реализующее эти методы.

\textbf{Цель работы} – ознакомиться с численными методами решения задач математической физики, решить предложенные типовые задачи, сформулировать выводы по полученным решениям, отметить достоинства и недостатки методов, приобрести практические навыки и компетенции, а также опыт самостоятельной профессиональной деятельности, а именно:
\begin{itemize}
    \item создать алгоритм решения поставленной задачи и реализовать его, протестировать программы;
    \item освоить теорию вычислительного эксперимента; современных компьютерных технологий; 
    \item приобрести навыки представления итогов проделанной работы в виде отчета, оформленного в соответствии с имеющимися требованиями, с привлечением современных средств редактирования и печати.
\end{itemize}

\noindent Работа над курсовым проектом предполагает выполнение следующих задач:
\begin{itemize}
    \item дальнейшее углубление теоретических знаний обучающихся и их систематизацию;
    \item получение и развитие прикладных умений и практических навыков по направлению подготовки;
    \item овладение методикой решения конкретных задач;
    \item развитие навыков самостоятельной работы;
    \item развитие навыков обработки полученных результатов, анализа и осмысления их с учетом имеющихся литературных данных;
    \item приобретение навыков оформления описаний программного продукта;
    \item повышение общей и профессиональной эрудиции.
\end{itemize}