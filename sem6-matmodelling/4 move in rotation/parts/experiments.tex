\section{Вычислительные эксперименты}

    \subsection{Алгоритм}
        Для реализации моделей сделаем замену переменных \( \beta(t) = \dot{\alpha}(t) \), и получим систему дифференциальных уравнений первого порядка:
        \[
            \begin{cases}
                & \dot{\beta} + w^2 \sin \alpha = 0, \\
                & \beta = \dot{\alpha}, \\
                & \beta(0) = \alpha_1, \quad \alpha(0) = \alpha_0.
            \end{cases} \Rightarrow
            \begin{cases}
                & \dot{\beta} = - w^2 \sin \alpha, \\
                & \dot{\alpha} = \beta, \\
                & \beta(0) = \alpha_1, \quad \alpha(0) = \alpha_0.
            \end{cases}
        \]
        Для системы данного вида можно применять различные численные методы. Уравнения с дополнительными силами аналогично приводятся к такому виду.

        Для компьютерного вычисления будем использовать метод Рунге-Кутты, с помощью которого получим численное решение системы дифференциальных уравнений с заданными параметрами. После чего построим их решения и фазовые плоскости.



    \subsection{Программа}
        Для расчётов и визуализации был использован язык Python с библиотеками numpy и matplotlib.

    \subsection{Результаты}
        