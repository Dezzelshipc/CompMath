\section{Математическая модель}
    Мы изучаем перемещение тела по по вращающемуся диску. Примем за тело материальную точку с массой \( m \), а угловую скорость диска -- \( \Omega \). Трением будем пренебрегать.

    Поскольку система является неинерциальной, то на тело действует сила инерции:
    \[
        \overrightarrow{F_i} = m \frac{d \overrightarrow{V}}{dt},
    \]
    где \( \overrightarrow{V} \) -- вектор скорости.

    Во вращающейся системе отсчёта наблюдателю кажется, что тела движутся по изогнутой траектории. Такой эффект называется эффектом Кориолиса. Сила Кориолиса равна:
    \[
        \overrightarrow{F_k} = m \frac{d \overrightarrow{V}}{dt},
    \]
    но в общем случае сила Кориолиса \( \overrightarrow{F_k} \perp \overrightarrow{V} \) и равна:
    \[
        \overrightarrow{F_k} = - 2 m \left[ \overrightarrow{\Omega} \times \overrightarrow{V} \right].
    \]
    Приравнивая формулы имеем:
    \[
        m \frac{d \overrightarrow{V}}{dt} = - 2 m \left[ \overrightarrow{\Omega} \times \overrightarrow{V} \right].
    \]
    Преобразовывая, мы получаем систему линейных дифференциальных уравнений:
    \[
        \left\{ \begin{split}
            & \ddot{x} = 2 \Omega \dot{y}, \\
            & \ddot{y} = -2\Omega \dot{x}.
        \end{split} \right.
    \]
    Уравнения второго порядка, значит нужно 2 начальных условия для каждого -- положение и скорость:
    \[
        \left\{\begin{split}
            & x(0) = x_0, & \dot{x}(0) = x_1, \\
            & y(0) = y_0, & \dot{y}(0) = y_1.
        \end{split}\right.
    \]