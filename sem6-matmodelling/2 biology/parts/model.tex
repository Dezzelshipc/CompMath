\section{Построение математической модели}
    Мы будем строить модель для двух популяций и рассматривать их взаимодействие во времени. Пусть $ N(t) $ -- популяция жертв, $ M(t) $ -- популяция хищников в зависимости от времени. Единицы измерения выберем безразмерные и скажем, что это некоторый <<объём>> популяции. В общем случае будем рассматривать при $ N(t), M(t) > 0 $. Время измеряется в секундах.

    В отсутствии взаимодействия между ними популяция хищников будет расти, а популяция жертв убывать в зависимости от объёмов самих популяций:
    \[
        \begin{cases}
            \dfrac{dN}{dt} = a \cdot N, \\[.5em]
            \dfrac{dM}{dt} = -b \cdot M,
        \end{cases}
    \]
    где $a, b > 0$ -- коэффициенты, обозначающие скорость изменения популяции.

    При взаимодействии популяции будут влиять друг на друга. Хищники будут уменьшать популяцию жертв, а жертвы будут увеличивать популяцию хищников:
    \[
        \begin{cases}
            \dfrac{dN}{dt} = (a - c \cdot M) N, \\[.5em]
            \dfrac{dM}{dt} = (-b + d \cdot N) M,
        \end{cases}
    \]
    где $ c, d > 0 $ -- коэффициенты, обозначающие скорость изменения популяции в зависимости от второй популяции.

    Таким образом, построили модель, являющуюся системой обыкновенных дифференциальных уравнений, которая описывает взаимодействие двух популяций модели <<Хищник-жертва>>. Для получения единственного решения добавим начальные условия: 
    \[
        \begin{cases}
            N(0) = N_0, \\[.5em]
            M(0) = M_0,
        \end{cases}
    \]