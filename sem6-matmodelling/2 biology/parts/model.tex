\section{Математическая модель}
    Мы будем строить модель для двух популяций и рассматривать их взаимодействие во времени. Пусть $ N(t) $ -- популяция жертв, $ M(t) $ -- популяция хищников в зависимости от времени. Единицы измерения выберем безразмерные и скажем, что это некоторый <<объём>> популяции за некоторый промежуток времени. Модель будет иметь физический смысл при $ N(t), M(t) > 0 $.

    В отсутствии взаимодействия между ними, популяция жертв будет размножаться. Чем больше популяция будет, тем больше она будет расти, а значит скорость роста зависит от объёма самой популяции. Популяция хищников, наоборот, будет умирать и убывать со временем. Чем меньше будет хищников, тем реже они будут умирать, и поэтому скорость изменения будет тоже зависит от объёма этой популяции:
    \[
        \begin{cases}
            \dfrac{dN}{dt} = a \cdot N, \\[.5em]
            \dfrac{dM}{dt} = -b \cdot M,
        \end{cases}
    \]
    где $a, b > 0$ -- коэффициенты, обозначающие скорость изменения популяции.

    При взаимодействии популяции будут влиять друг на друга. Скорость изменения будет зависеть уже от объёма обоих популяций, потому что чем больше хищников, тем быстрее они будут поглощать жертв, и чем больше жертв, тем больше их могут находить и поглощать хищники. Соответственно объём популяции жертв будет уменьшаться, а хищников увеличиваться пропорционально обоим популяциям.
    \[
        \begin{cases}
            \dfrac{dN}{dt} = (a - c \cdot M) N, \\[.5em]
            \dfrac{dM}{dt} = (-b + d \cdot N) M,
        \end{cases}
    \]
    где $ c, d > 0 $ -- коэффициенты, обозначающие скорость изменения популяции в зависимости от второй популяции.

    Таким образом, построили модель конкуренции, которая является системой двух обыкновенных дифференциальных уравнений. Для получения единственного решения добавим начальные условия: 
    \[
        \begin{cases}
            N(0) = N_0, \\[.5em]
            M(0) = M_0.
        \end{cases}
    \]