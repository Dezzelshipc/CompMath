\section{Анализ модели}
    Исследуем дифференциальное уравнение на устойчивость.
    \[
        \frac{dT}{dt} = 0 \Rightarrow  P - k S (T - T_0) - \sigma S (T^4 - T_0^4) = 0,
    \]
    \[
        \sigma T^4 + k T - \sigma T_0^4 - k T_0 - \frac{P}{S} = 0.
    \]
    Воспользуемся матрицей Гурвица для определения положительности корней.
    \[
        a_0 = \sigma, ~ a_{1,2} = 0, ~ a_3 = k, ~ a_4 = - \sigma T_0^4 - k T_0 - \frac{P}{S} 
    \]
    \[
        \left(\begin{matrix}
            a_1 & a_3 & 0 & 0 \\
            a_0 & a_2 & a_4 & 0 \\
            0 & a_1 & a_3 & 0 \\
            0 & a_0 & a_2 & a_4
        \end{matrix}\right) 
        =
        \left(\begin{matrix}
            0 & k & 0 & 0 \\
            \sigma & 0 & a_4 & 0 \\
            0 & 0 & k & 0 \\
            0 & \sigma & 0 & a_4
        \end{matrix}\right) 
    \]
    Рассчитаем главные миноры: \( M_1 = 0, ~ M_2 = - k \sigma, ~ M_3 = k M_2, ~ M_4 = a_4 M_3 \). Заметим, что существует отрицательный минор, значит существует положительный корень.

    Также заметим, что удельная теплоёмкость и масса влияют только на скорость увеличения температуры, но не на точки равновесия.

    \subsection{Анализ конкретной модели}
        Поскольку анализ модели в общем случае является непростым  в связи с 4 степенью параметра, проанализируем на примере конкретной модели. Для этого примени метод анализа по первому приближению. Возьмём параметры (Данная модель будет построена в разделе <<Вычислительные эксперименты>>):
        \[
            P = 3000 \text{Вт}, ~ m = 0.5 \text{кг}, ~ c = 897 \frac{\text{Дж}}{\text{кг} \cdot \text{К}}, ~ S = 0.4 \text{м}^2, ~ k = 2, ~ T_0 = 296 \text{K}.
        \]
        Найдём точки устойчивости модели с данными параметрами:
        \[
            T_1 = -645.06\dots, ~
            T_2 = 599.58\dots, ~
            T_{3,4} = 22.73\dots \pm i 623.15\dots.
        \]
        Исследуем вещественные нули. Для этого найдём производную правой части и подставим данные значения (обозначим правую часть дифференциального уравнения $R$):
        \[
            \left.\frac{dR}{dT}\right|_{T_1} = 0.053\dots, \quad
            \left.\frac{dR}{dT}\right|_{T_2} = -0.045\dots
        \]
        Для $T_1$ получили положительное значение, значит данное положение равновесия неустойчивое. Для $T_2$ получили отрицательное, значит оно устойчивое.
