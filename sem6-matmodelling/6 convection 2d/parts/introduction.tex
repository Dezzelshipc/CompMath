\section{Введение}
    ИЗ ДРУГОГО
    Как говорил Альберт Эйнштейн: <<Всё в мире относительно>>. И действительно, есть много разных точек зрения на одно и то же явление, и физика не является исключением. Например, когда мы едем в автобусе, мы перемещаемся относительно земли, но находимся на месте в самом автобусе. На космическом уровне Земля перемещается одним путём относительно Солнца, и другим относительно центра галактики.

    Интересным представляет собой перемещение во вращающемся объекте, например, человека на карусели, или перемещение около полюса земли.Изучим, как происходит перемещение со стороны человека вне вращающейся системы координат.
