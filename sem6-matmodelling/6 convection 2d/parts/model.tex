\section{Математическая модель}
    В общем случае для процессов распространения используется уравнение переноса:
    \[
        \frac{\partial C}{\partial t} + u \frac{\partial C}{\partial x} + v \frac{\partial C}{\partial y} = 0,
    \]
    где \( t \) -- время, \( x, y \) -- координаты, \( C \) -- концентрация вещества (или, например, температура) в каждой точке пространства, \( u, v \) -- компоненты скорости течения по \( x \) и \( y \). 
    
    При этом в начальный момент времени известна концентрация
    \[
        C(x, y, 0) = C_0 (x, y).
    \]
    
    Также задана функция тока \( \psi(x, y) \), задающая перемещение примесей: 
    \[
        \left\{
            \begin{split}
                & u(x, y) = -\frac{\partial \psi}{\partial y}, \\
                & v(x, y) = \frac{\partial \psi}{\partial x}.
            \end{split}
        \right.
    \]

    Задача состоит в том, чтобы найти концентрацию вещества в каждой точке области в каждый момент времени.

    Однако, необязательно знать дифференциальное уравнение, которое задаёт процесс, достаточно знать функцию тока. В этом поможет метод частиц, также известный как метод Лагранжа.

    Для этого метода нужно моделировать движение каждой точки отдельно. Рассмотрим это на примере одной частицы. Она имеет координаты \( x, y \) и концентрацию \( C \), которая не меняется со временем \( \left( \frac{\partial C}{\partial t} = 0 \right) \). В данной точке на неё действует ток: на координату \( x \) действует \( u(x, y) \), на \( y \) действует \( v(x, y) \). Аппроксимируя, мы можем записать изменение координат точки от момента времени \( n \) до \( n+1 \) (через время \( \Delta t \)):
    \[
        \left\{
            \begin{split}
                & x^{n+1} = x^n + u(x^n, y^n) \Delta t, \\
                & y^{n+1} = y^n + v(x^n, y^n) \Delta t.
            \end{split}
        \right.
    \]
    Совершая предельный переход при \( \Delta t \to 0 \), можем записать систему дифференциальных уравнений для частицы:
    \[
        \left\{
            \begin{split}
                & \frac{dx}{dt} = u(x, y), \\
                & \frac{dy}{dt} = v(x, y).
            \end{split}
        \right.
    \]

    Применяя данную схему для большого количества частиц мы можем получить приближённую функцию концентрации в каждой точке, например, интерполированием.