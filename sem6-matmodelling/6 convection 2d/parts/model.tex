\section{Математическая модель}
    В общем случае для процессов распространения используется уравнение переноса:
    \[
        \frac{\partial C}{\partial t} + u \frac{\partial C}{\partial x} + v \frac{\partial C}{\partial y} = 0,
    \]
    где \( t \) -- время, \( x, y \) -- координаты, \( C \) -- концентрация вещества (или, например, температура) в каждой точке пространства, \( u, v \) -- компоненты скорости течения по \( x \) и \( y \). 
    
    При этом в начальный момент времени известна концентрация
    \[
        C(x, y, 0) = C_0 (x, y),
    \]
    и она не будет меняться со временем \( \frac{\partial C}{\partial t} = 0 \). Также задана функция тока \( \psi(x, y) \), которая задаёт перемещение примесей: 
    \[
        \left\{
            \begin{split}
                & u(x, y) = -\frac{\partial \psi}{\partial y}, \\
                & v(x, y) = \frac{\partial \psi}{\partial x}.
            \end{split}
        \right.
    \]

    Однако, необязательно знать дифференциальное уравнение, которое задаёт процесс, достаточно знать функцию тока. В этом поможет метод частиц, также известный как метод Лагранжа.

    Для этого метода нужно моделировать движение каждой точки отдельно. Рассмотрим это на примере одной частицы. Она имеет координаты \( x, y \). В данной точке на неё действует ток: на координату \( х \) действует \( u(x, y) \), на \( y \) действует \( v(x, y) \). Поэтому можем составить систему дифференциальных уравнений для частицы:
    \[
        \left\{
            \begin{split}
                & \frac{dx}{dt} = u(x, y), \\
                & \frac{dy}{dt} = v(x, y).
            \end{split}
        \right.
    \]

    Применяя данную схему для большого количества частиц мы можем получить приближённую функцию концентрации в каждой точке, например, интерполированием.