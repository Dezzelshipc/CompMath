\section{Вычислительные эксперименты}
    Исследуемая область является квадратом, размером 1 на 1. В качестве функции тока возьмём:
    \[
        \psi(x, y) = \sin (2 \pi x) \sin(\pi y),
    \]

    откуда получаем:
    \[
        \left\{
            \begin{split}
                & u(x, y) = -\pi \sin (2 \pi x) \cos(\pi y), \\
                & v(x, y) = 2\pi \cos (2 \pi x) \sin(\pi y).
            \end{split}
        \right.
    \]

    Изобразим фазовые кривые этой функции тока (Рис. \ref{streamplot}).
    \begin{figure}[H]
        \centering
        \includegraphics[width=8cm]{pictures/streamplot.pdf}
        \includegraphics[width=8cm]{pictures/quiver.pdf}
        \caption{Фазовые кривые функции тока \( \psi(x, y) \).} \label{streamplot}
    \end{figure}

    В качестве начальной концентрации используем такую функцию:
    \[
        C_0(x, y) = \arctan \left( \frac{y - 0.5}{0.1} \right)
    \]

    \subsection{Алгоритм}
        Для реализации модели будем использовать конечно-разностный метод Рунге-Кутты 4 порядка. Для этого был использован язык Python с библиотеками numpy, matplotlib и scipy.

        Создадим в исследуемой области большое количество точек и для каждой из них применим метод на промежутке времени \( [0, 0.4] \). Выберем несколько моментов времени в которых построим все точки, а также интерполяционное изображение из этих точек.
        
        Интерполяция будет производиться методом линейной триангуляции, который реализуется встроенной функцией griddata.

    \subsection{Численный анализ}
        Для начала построим изображение пути трёх точек в данной области (Рис. \ref{3pts}).
        \begin{figure}[H]
            \centering
            \includegraphics[width=12cm]{pictures/three.pdf}
            \caption{Пример движения трёх случайных точек} \label{3pts}
        \end{figure}
        Эти пути следуют фазовым кривым, чего и следовало ожидать.

        Теперь рассмотрим поведение только одной центральной линии из 2000 точек на меньшем промежутке времени.  
        \begin{figure}[H]
            \centering
            \includegraphics[width=8cm]{pictures/line0.pdf}
            \includegraphics[width=8cm]{pictures/line3.pdf}
            \includegraphics[width=8cm]{pictures/line6.pdf}
            \includegraphics[width=8cm]{pictures/line9.pdf}
            \caption{Движение одной линии.} \label{line}
        \end{figure}

        На этом примере (Рис. \ref{line}) можно сделать предположения о поведении большого множества точек в данной области, например, образование изображения, похожего на гриб.


    \subsection{Результаты}
        Для получения результата на всей области построим 20000 случайных точек, после чего смоделируем их движение. После чего интерполируем на сетке, состоящей из 500 делений по обоим направлениям.

        Слева находится изображение со всеми точками, а справа интерполированное изображение в этот же момент времени.

    \begin{figure}[H]
        \centering
        \includegraphics[width=8cm]{pictures/s0.png}
        \includegraphics[width=8cm]{pictures/p0.pdf}
    \end{figure}
    \begin{figure}[H]
        \centering
        \includegraphics[width=8cm]{pictures/s5.png}
        \includegraphics[width=8cm]{pictures/p5.pdf}
    \end{figure}
    \begin{figure}[H]
        \centering
        \includegraphics[width=8cm]{pictures/s10.png}
        \includegraphics[width=8cm]{pictures/p10.pdf}
    \end{figure}
    \begin{figure}[H]
        \centering
        \includegraphics[width=8cm]{pictures/s15.png}
        \includegraphics[width=8cm]{pictures/p15.pdf}
    \end{figure}
    \begin{figure}[H]
        \centering
        \includegraphics[width=8cm]{pictures/s20.png}
        \includegraphics[width=8cm]{pictures/p20.pdf}
        \caption{Результаты вычисления для 20000 точек.} \label{res1}
    \end{figure}

    Численные результаты (Рис. \ref{res1}) показывают образование чего-то похожего на гриб, как и было предположено в анализе. 
    
    В интерполированных результатах около границ можно заметить некоторые разрывы. Поскольку это модель <<идеальная>>, то есть без какого-либо дополнительного влияния это не может быть турбулентность. Если посмотреть на результат, построенный из точек, то можно увидеть почему так происходит: близко к границам есть точки с совершенно разной концентраций. Метод линейной триангуляции выбирает 3 точки и интерполирует значение в области между ними, а поскольку точки расположены очень плотно, то области между ними маленькие, а значения меняются быстро.


    \subsection{Турбулентность}
        В реальном мире очень сложно добиться идеальных условий где-либо, и для жидкостей и газов это явление называется турбулентностью. Для приближённого моделирования мы можем добавить некоторою случайную величину в дифференциальное уравнение, которая заставит частицу немного отклониться от траектории, заданной током.    
        \begin{figure}[H]
            \centering
            \includegraphics[width=8cm]{pictures/pr0.pdf}
            \includegraphics[width=8cm]{pictures/pr5.pdf}
            \includegraphics[width=8cm]{pictures/pr10.pdf}
            \includegraphics[width=8cm]{pictures/pr15.pdf}
            \includegraphics[width=8cm]{pictures/pr20.pdf}
            \caption{Результаты при добавлении случайной величины.} \label{turb}
        \end{figure}

        В этом эксперименте добавлена слагаемым случайная величина из равномерного распределения на отрезке \( [-0.5, 0.5] \). 
        Видно (Рис. \ref{turb}), что в отличие от гладкого изображения идеальной модели здесь наблюдаются искажения из-за турбулентности.