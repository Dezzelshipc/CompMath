\section{Математическая модель}
    Движение математического маятника во времени можно описать на Декартовой плоскости $ (x, y) $ в зависимости от времени \( t \). Предположим, что сам маятник является материальной точкой, длина невесомой нити \( L \) (м) и ускорение свободного падения \(\left( g \approx 9.8 \dfrac{\text{м}}{\text{с}^2} \right) \) постоянны. Тогда движение маятника вместо пары \((x, y) \) можно описать углом отклонения от вертикальной оси \( \alpha \).

    Воспользуемся уравнением моментов для материальной точки:
    \[
        J \dfrac{d^2 \alpha}{dt^2} = M,
    \]
    где $ J $ -- момент инерции относительно оси, $ M $ -- момент сил. Для материальной точки: \( J = mL^2, \) где $ m $ -- масса маятника. 

    На маятник влияет сила тяжести, равная \( F = mg \), но на движение будет влиять только её составляющая, касательная к движению, поэтому \( F = -mg \sin \alpha \). Эта сила перпендикулярна к нити, поэтому момент силы равен:
    \[
        M = FL = - mgL \sin \alpha.
    \]

    Подставляя в уравнение моментов, и приводя слагаемые:
    \[
        mL^2 \dfrac{d^2 \alpha}{dt^2} + mgL \sin \alpha = 0 \Rightarrow \dfrac{d^2 \alpha}{dt^2} + \dfrac{g}{L} \sin \alpha = 0.
    \]

    Получили дифференциальное уравнение второго порядка, которое описывает угол отклонения маятника в зависимости от времени.
    
    Обозначим \( w^2 = \dfrac{g}{L} \) и добавим начальные условия для получения единственного решения:
    \[
        \begin{cases}
            & \ddot{\alpha} + w^2 \sin \alpha = 0, \\
            & \dot{\alpha}(0) = \alpha_1, \quad \alpha(0) = \alpha_0.
        \end{cases}
    \]
    
    Известно, что при небольших значениях углов \( \sin x \approx x \), поэтому из нелинейной модели сделаем линейную: \( \ddot{\alpha} + w^2 \alpha = 0. \)

    \subsection{Модели с внешними силами}
        Одной из внешних сил является трение. Оно зависит от скорости с некоторым коэффициентом $ k > 0 $. Получим модель с трением: \( \ddot{\alpha} + k \dot{\alpha} + w^2 \alpha = 0. \)

        Также внешними силами могут быть вынуждающие колебания: \( A_f \sin \left( w_f t \right) \) с амплитудой \( A_f \) и частотой \( w_f \). Модель с вынуждающими колебаниями:\\ \( \ddot{\alpha} + w^2 \alpha = A_f \sin \left( w_f t \right). \)

        Если на маятник будут действовать сразу обе предыдущие силы, то уравнение будет выглядеть так: \( \ddot{\alpha}  + k \dot{\alpha} + w^2 \alpha = A_f \sin \left( w_f t \right). \)