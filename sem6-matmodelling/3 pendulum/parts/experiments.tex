\section{Вычислительные эксперименты}

    \subsection{Алгоритм}
        Для компьютерного вычисления будем использовать метод Рунге-Кутты, с помощью которого получим численное решение системы дифференциальных уравнений с заданными параметрами. После чего построим графики решений на координатной и фазовой плоскостях.

        Для модели с силой трения и вынужденными колебаниями для каждой частоты колебаний найдём амплитуду


    \subsection{Программа}
        Для расчётов и визуализации был использован язык Python с библиотеками numpy и matplotlib.

        \lstinputlisting[language=Python,
        captionpos=t,
        style=colored,
        basicstyle=\footnotesize\dejavu,
        frame=lines]{src/3model.py}

    \subsection{Результаты}