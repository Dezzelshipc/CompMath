\section{Анализ модели}
    Найдём точки равновесия системы.
    \[
        \begin{cases}
            \dfrac{dN}{dt} = 0, \\[.5em]
            \dfrac{dM}{dt} = 0;
        \end{cases}
        \Rightarrow
        \begin{cases}
            (a - c \cdot M) N = 0, \\[.5em]
            (-b + d \cdot N) M = 0;
        \end{cases}
        \Rightarrow
        \begin{cases}
            M_e = \dfrac{a}{c}, \\[.5em]
            N_e = \dfrac{b}{d}.
        \end{cases}
    \]
    Также есть тривиальное решение $ (0, 0) $.

    Применим метод первого приближения. Найдём матрицу Якоби, подставим точки равновесия и найдём собственные значения матрицы, после чего по ним укажем тип устойчивости.
    Матрица Якоби:
    \[
        J = \left(\begin{matrix}
            a - c \cdot M & -c \cdot N \\
            d \cdot M & -b + d \cdot N
        \end{matrix}\right).
    \]
    Подставим точки равновесия в матрицу и найдём собственные значения.
    \[
        J\big|_{(0,0)} = \left(\begin{matrix}
            a & 0\\
            0 & -b
        \end{matrix}\right)
        \Rightarrow
        \lambda_1 = -b < 0, ~ \lambda_2 = a > 0.
    \]
    Нулевая точка является седловой точкой, поскольку оба собственных значения вещественны и разных знаков.
    \[
        J\big|_{(N_e,M_e)} = \left(\begin{matrix}
            0 & -\dfrac{c \cdot b}{d}\\
            \dfrac{d \cdot a}{c} & 0
        \end{matrix}\right)
        \Rightarrow
        \lambda_{1,2} = \pm i \sqrt{ab}.
    \]
    Точка $ (N_e, M_e) $ является неасимптотически устойчивой, поскольку собственные значения полностью мнимые. Это значит, что в самой точке не будет происходить изменения величин, но на удалении от неё будут находиться циклы.

    Теперь найдём первый интеграл системы. Для этого поделим первое уравнение на второе и разделим переменные:
    \[
        \frac{dN}{dM} = \frac{(a - c \cdot M)N}{(-b + d \cdot N)M} \Rightarrow
        \left( \frac{a}{M} - c \right)dM + \left( \frac{b}{N} - d \right)dN = 0.
    \]
    Из чего интегрированием получаем:
    \[
        a \ln(M) - c \cdot M + b \ln(N) - d \cdot N = const.
    \]
    Значит данное соотношение величин не будет изменяться с течением времени. Но мы знаем, что объёмы популяций будут меняться, а значит в какой-то момент времени вернутся к начальным значениям.