\section{Анализ модели}
    Проанализируем линейную модель:
    \[
        \begin{cases}
            & \ddot{\alpha} + w^2 \alpha = 0, \\
            & \dot{\alpha}(0) = \alpha_1, \quad \alpha(0) = \alpha_0.
        \end{cases}
    \]

    Известен вид аналитического решения: \( \alpha(t) = C_1 \cos(wt) + C_2 \sin(wt) \). Находим частное решение: \( \alpha(t) = \alpha_0 \cos(wt) + \dfrac{a_1}{w} \sin(wt) \). Можно найти такой угол $\varphi$ и константу $\rho$, что \( \rho\sin \varphi = \alpha_0 \) и \( \rho\cos \varphi = \dfrac{a_1}{w} \), поэтому по формуле синуса суммы: \( \alpha(t) = \rho\sin(wt + \varphi) \).

    Значит результатом решения будет синусоида с некоторым смещением в зависимости от начальных параметров.