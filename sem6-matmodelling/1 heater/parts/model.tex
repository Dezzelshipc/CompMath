\section{Построение математической модели}
    Главной характеристикой любого нагревателя является температура. При включении нагревателя температура со временем растёт. Значит нужно найти зависимость температуры от времени: $ T(t) $, где $[T] = $ K, $ [t] =$ сек.

    Во время процесса нагревания изменяется количество теплоты тела на $ \Delta Q $ (Дж). Его можно выразить формулой:
    \[
        \Delta Q = cm \Delta T,
    \]
    где $c$ -- удельная теплоёмкость тела $\left(\frac{\text{Дж}}{\text{кг} \cdot \text{К}}\right)$, $m$ -- масса тела (кг), $ \Delta T $ - изменение температуры.

    С другой стороны, поскольку наш нагревательный прибор работает от электричества, можно выразить количество теплоты иначе:
    \[
        \Delta Q = P \Delta t,
    \]
    где $P$ -- мощность (Вт), $\Delta t$ -- изменение времени. 

    Предположим, что окружающая температура постоянная и равна $T_0$, и поэтому будет происходить охлаждение, в зависимости от площади и общей конструкции нагревателя. Добавим слагаемое: $ -k S (T - T_0) \Delta t $, где $S$ -- площадь ($\text{м}^2$), $k$ - коэффициент, который зависит от конструкции.

    Также будем учитывать тепловое излучение, которое происходит в результате нагревания, используя закон Стефана--Больцмана: $ -\sigma S (T^4 - T_0^4) \Delta t $, где $\sigma \approx 5.68 \cdot 10^{-8} \frac{\text{Вт}}{\text{м}^2 \text{К}^4}$ -- постоянная Стефана--Больцмана.

    В итоге получаем:
    \[
        cm\Delta T = P \Delta t - k S (T - T_0) \Delta t - \sigma S (T^4 - T_0^4) \Delta t.
    \]
    Делим обе части на $ cm \Delta t $ и совершаем предельный переход при $ \Delta t \to 0 $:
    \[
        \frac{dT}{dt} = \frac{ P - k S (T - T_0) - \sigma S (T^4 - T_0^4) }{cm}
    \]
    Получили дифференциальное уравнение, которое описывает поведение температуры нагревателя.

    \subsection{Модель с терморегулятором}
    В реальном мире целесообразно ограничить максимальную температуру. Для этого введём функцию <<переключатель>>, которая по достижении максимальной температуры $ T_{max} $ отключит нагреватель, и после чего по достижении температуры включения $ T_{min} $ снова включит его.
    \[
        H(T, T_{max}, T_{min}) = \left\lbrace \begin{matrix}
            0, T > T_{max},\\
            1, T < T_{min}.
        \end{matrix}\right.
    \]
    Добавляя в уравнение:
    \[
        \frac{dT}{dt} = \frac{ P \cdot H(T, T_{max}, T_{min}) - k S (T - T_0) - \sigma S (T^4 - T_0^4) }{cm}
    \]