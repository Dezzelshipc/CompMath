\section{Построение математической модели}
    Главной характеристикой любого нагревателя является его температура. При включении нагревателя со временем температура изменяется. Поэтому нужно найти зависимость температуры (К) от времени (с): $ T(t) $.

    Сделаем предположение, что нагревательный элемент состоит из одного материала и окружающая температура постоянная и равна $T_0$.% ??
    
    Процесс нагревания можно описать уравнением теплового баланса. Изменение внутренней энергии тела на $ \Delta Q $ (Дж) описывается формулой:
    \[
        \Delta Q = cm \Delta T,
    \]
    где $c$ -- удельная теплоёмкость тела $\left(\frac{\text{Дж}}{\text{кг} \cdot \text{К}}\right)$, $m$ -- масса тела (кг), $ \Delta T $ - изменение температуры.

    Поскольку наш нагревательный прибор работает от электричества, он потребляет мощность во время работы, за счёт чего изменяет свою внутреннюю энергию:
    \[
        \Delta Q_1 = P \Delta t,
    \]
    где $P$ -- мощность (Вт), $\Delta t$ -- изменение времени. 

    На внутреннюю энергию также влияют входящие и исходящие тепловые потоки. На единицу площади за единицу времени исходящий поток изменяет энергию на $ -k T $, а входящий на $ k T_0 $, где $ k > 0$ - коэффициент, который зависит от конструкции. Учитывая тепловые потоки, внутрення энергия изменяется на
    \[
        \Delta Q_2 = -k S (T - T_0) \Delta t,
    \]
    где $S$ -- площадь нагревателя ($\text{м}^2$).

    Также любое тело, нагретое выше абсолютного нуля, начинает изучать, что описывает закон Стефана-Больцмана. На единицу площади за единицу времени нагреватель излучает энергию равную $ -\sigma T^4 $, а изучение из внешней среды изменяет энергию на $ \sigma T_0^4 $, где $\sigma \approx 5.68 \cdot 10^{-8} \frac{\text{Вт}}{\text{м}^2 \text{К}^4}$ -- постоянная Стефана--Больцмана. Значит, общее изменение энергии за счёт излучения:
    \[
        \Delta Q_3 = -\sigma S (T^4 - T_0^4) \Delta t.
    \]


    В итоге, применяя закон теплового баланса, получаем:
    \[
        cm\Delta T = P \Delta t - k S (T - T_0) \Delta t - \sigma S (T^4 - T_0^4) \Delta t.
    \]
    Делим обе части на $ cm \Delta t $ и совершаем предельный переход при $ \Delta t \to 0 $:
    \[
        \frac{dT}{dt} = \frac{ P - k S (T - T_0) - \sigma S (T^4 - T_0^4) }{cm}.
    \]
    Получили дифференциальное уравнение, которое описывает поведение температуры нагревателя. Для получения единственного решения добавим начальное условие: \( T(0) = T_0 \).

    \subsection{Модель с терморегулятором}
        В быту, для того чтобы нагреватель не нагревался до опасных температур, целесообразно ограничить максимальную температуру. Для этого введём функцию <<переключатель>>, которая по достижении максимальной температуры $ T_{max} $ отключит нагреватель, и после чего по достижении температуры включения $ T_{min} $ снова включит его.
        \[
            H(T, T_{max}, T_{min}) = \begin{cases}
                0, T > T_{max},\\
                1, T < T_{min}.
            \end{cases}
        \]
        Добавляя в уравнение:
        \[
            \frac{dT}{dt} = \frac{ P \cdot H(T, T_{max}, T_{min}) - k S (T - T_0) - \sigma S (T^4 - T_0^4) }{cm}.
        \]