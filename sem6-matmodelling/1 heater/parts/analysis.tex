\section{Анализ модели}
    Найдём точки равновесия дифференциального уравнения.
    \[
        \frac{dT}{dt} = 0 \Rightarrow  \frac{P - k S (T - T_0) - \sigma S (T^4 - T_0^4)}{cm} = 0.
    \]
    Заметим, что удельная теплоёмкость и масса находятся в знаменателе, а значит не влияют на нули, однако они влияют на скорость изменения температуры.
    \[
        T^4 + \frac{k}{\sigma} T - \left( T_0^4 + \frac{k T_0 + \frac{P}{S}}{\sigma} \right) = 0.
    \]
    Уравнение четвёртой степени, а значит оно имеет ровно 4 корня на \(\mathbb{C}\). Поставим задачу определить тип этих корней: их положительность или комплексность.
    
    Для удобства переобозначим: 
    \[
            a = \frac{k}{\sigma} > 0, \quad b = \left( T_0^4 + \frac{k T_0 + \frac{P}{S}}{\sigma} \right) > 0, \quad T^4 + a T - b = 0.
    \]
    Воспользуемся теоремой Декарта: <<Число положительных корней многочлена с вещественными коэффициентами равно числу перемен знаков в ряду его коэффициентов или на чётное число меньше этого числа>>. Знак коэффициентов нашего уравнения меняется только раз -- между последними двумя, значит существует ровно один положительный корень. Подставляя в уравнение $ T = -T $, найдём количество отрицательных корней. 
    \[ \quad T^4 - a T - b = 0. \]
    Знак меняется также один раз, значит существует ровно один отрицательный корень.

    Из предыдущего следует, что остаётся два комплексных корня. Покажем это. Данное уравнения можно свести к кубическому уравнению разольвенты
    \[
        x^4 + px^2 + qx + r = 0 \Rightarrow y^3 -2py^2 + (p^2 - 4r)y + q^2 = 0,
    \]
    корни которой связаны с корнями исходного уравнения
    \[
        y_1 = (x_1 + x_2)(x_3+x_4), ~ y_2 = (x_1 + x_3)(x_2+x_4), ~ y_3 = (x_1 + x_4)(x_2+x_3).
    \]
    Сведём наше уравнения к разольвенте
    \[
        y^3 + 4by + a^2 = 0.
    \]
    Данное уравнение представлено в виде $(y^3 + py + q = 0)$, к которому можно применить формулу Кардано, а более конкретно, найти величину $ Q $, которая определит типы корней.
    \[
        Q = \left( \frac{p}{3} \right)^3 + \left( \frac{q}{2} \right)^2 = \left( \frac{4b}{3} \right)^3 + \left( \frac{a^2}{2} \right)^2.
    \]
    Поскольку $Q > 0$, то у уравнения один вещественный корень и два сопряжённых комплексных. 

    Из того, что среди $ y_i $ есть комплексный корень, следует, что и среди корней $ T_i $ тоже есть комплексный. Известно, что комплексные корни многочленов с вещественными коэффициентам всегда образуют комплексно-сопряжённые пары, значит, что среди $ T_i $ есть комплексно-сопряжённая пара.

    В итоге получили, что уравнение имеет один положительный, один отрицательный и пару комплексно-сопряжённых корней.

    Исследуем устойчивость. Обозначим правую часть дифференциального уравнения за $ R $.
    \[
        \frac{dR}{dT} = \frac{-kS - 4\sigma S T^3}{cm} = 0 \Rightarrow T_e = -\sqrt[3]{\frac{k}{4 \sigma}},
    \]
    В точке $ T_e $ функция достигает экстремума, при этом меньше этой точки всегда возрастает, а больше -- убывает. Отсюда следует, что это максимум. Но мы уже знаем, что многочлен всегда имеет два вещественных корня, соответственно, отрицательный находится левее $ T_e $, а значит производная положительная, а положительный корень правее $ T_e $ с отрицательной производной. Из чего, методом первого приближения, мы находим, что отрицательный корень неустойчивый, а положительный устойчивый. 

    В итоге получаем, что данное дифференциальное уравнение имеет пару комплексно-сопряжённых точек равновесия и пару вещественных, одно из которых всегда положительное и устойчивое, а второе отрицательное и неустойчивое.