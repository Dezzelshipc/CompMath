\subsection{Математическая модель}
    В модели Колмогорова для начала мы отказываемся от явного выражения функциональных зависимостей, ограничиваясь некоторыми качественными предположениями.

    Первое предположение состоит в том, что в популяциях хищников отсутствует внутривидовая конкуренция. 
    \[
        \left\{\begin{split}
            & \dot{x_1} = \varepsilon(x_1)x_1 - V_{12}(x_1)x_2 - V_{13}(x_1)x_3, \\
            & \dot{x_2} = K_{12}(x_1)x_2 - V_{23}(x_2)x_3, \\
            & \dot{x_3} = K_{13}(x_1)x_3 + K_{23}(x_2)x_3. 
        \end{split}\right.
    \]
    Сформулируем остальные предположения:
    \begin{enumerate}
        \item \( \varepsilon' < 0; ~ \varepsilon(0) > \varepsilon(\bar{x}_1) = 0 > \varepsilon(\infty)\). Здесь у жертв ограниченное количество ресурса и за него существует конкуренция. Поэтому без хищников прирост жертв с увеличением их количества в некоторый момент прекратится и стабилизируется на уровне \( \bar{x}_1 \).
        \item \( K'_{ij} > 0; ~ K_{ij}(0) < K_{ij}(x_i^*) = 0 < K_{ij} (\infty) \). Это значит, что при увеличении численности жертв коэффициент естественного прироста хищников возрастает. Коэффициент переходит от отрицательных значения при недостатке пищи к положительным.
        \item \( V_{ij}(0) = 0; ~ V_{ij} (x_i) > 0, x_i > 0 \). Этот коэффициент показывает количество жертв, поглощаемых одним хищником.
    \end{enumerate}

    
    Имеем автономную систему \( \dot{x} = f(x) \).