\subsection{Математическая модель}
    Моделью Колмогорова при одной популяции жертвы \(x (t)\) и одной хищника \(y (t)\) называется такая модель \cite{svilog}:
    \[
        \left\{\begin{split}
            & \dot{x} = \alpha(x)x - V(x) y, \\
            & \dot{y} = K(x) y,
        \end{split}\right.
    \]
    с такими предположениями:
    
    \begin{enumerate}
        \item В популяции хищников отсутствует внутривидовая конкуренция.
        \item \( \alpha' < 0; ~ \alpha(0) > 0 > \alpha(\infty) \). В отсутствии хищников прирост жертв с увеличением популяции уменьшается до критического момента.
        \item \( K' > 0; ~ K(0) < 0 < K(\infty) \). Коэффициент прироста хищников.
        \item \( V(x) > 0, x > 0; ~ V(0) = 0 \). Коэффициент поглощения жертв.
    \end{enumerate}

    Адаптируем данную модель для нашей схемы взаимодеиствия популяций жертвы и хищников и подробнее опишем качественные предположения о функциях:
    \[
        \left\{\begin{split}
            & \dot{x_1} = \varepsilon(x_1)x_1 - V_{12}(x_1)x_2 - V_{13}(x_1)x_3, \\
            & \dot{x_2} = K_{12}(x_1)x_2 - V_{23}(x_2)x_3, \\
            & \dot{x_3} = K_{13}(x_1)x_3 + K_{23}(x_2)x_3. 
        \end{split}\right.
    \]
    Сформулируем предположения:
    \begin{enumerate}
        \item \( \varepsilon' < 0; ~ \varepsilon(0) > \varepsilon(\bar{x}_1) = 0 > \varepsilon(\infty)\). Здесь у жертв ограниченное количество ресурса и за него существует конкуренция. Поэтому без хищников прирост жертв с увеличением их количества в некоторый момент прекратится и стабилизируется на уровне \( \bar{x}_1 \).
        \item \( K'_{ij} > 0; ~ K_{ij}(0) < K_{ij}(x_i^*) = 0 < K_{ij} (\infty) \). Это значит, что при увеличении численности жертв коэффициент естественного прироста хищников возрастает. Коэффициент переходит от отрицательных значения при недостатке пищи к положительным.
        \item \( V_{ij}(0) = 0; ~ V_{ij} (x_i) > 0, x_i > 0 \). Этот коэффициент показывает количество жертв, поглощаемых одним хищником.
    \end{enumerate}

    Имеем автономную систему \( \dot{x} = f(x) \).

    Будем исследовать систему с линейными функциями, которые удовлетворяют указанным предположениям.
    \[
        \begin{split}
            & \varepsilon(x) = -\varepsilon \cdot x + \delta, & ~ \varepsilon, \delta > 0, \\
            & K_{ij} (x) = k_{ij} \cdot x - m_{ij}, & ~ k_{ij}, m_{ij} > 0, \\
            & V_{ij} (x) = v_{ij} \cdot x, & ~ v_{ij} > 0. 
        \end{split}
    \]

    Тогда система будет выглядеть так:
    \[
        \left\{\begin{split}
            & \dot{x_1} = \left( -\varepsilon x_1 + \delta \right)x_1 - v_{12} x_1 x_2 - v_{13} x_1 x_3, \\
            & \dot{x_2} = \left( k_{12} x_1 - m_{12} \right)x_2 - v_{23} x_2 x_3, \\
            & \dot{x_3} = \left( k_{13} x_1 - m_{13} \right)x_3 + \left( k_{23} x_2 - m_{23} \right)x_3. 
        \end{split}\right.
    \]