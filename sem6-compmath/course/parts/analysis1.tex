\subsection{Анализ модели}
    Матрица Якоби для данной системы:
    \[
        J = \frac{\partial f}{\partial x} = \left(\begin{matrix}
            \varepsilon_1 - \alpha_{12} x_2 - \alpha_{13} x_3 & -\alpha_{12} x_1 & -\alpha_{13} x_1 \\
            k_{12} \alpha_{12} x_2 & \varepsilon_2 + k_{12} \alpha_{12} x_1 - \alpha_{23} x_3 & -\alpha_{23} x_2 \\
            k_{13} \alpha_{13} x_3 & k_{23} \alpha_{23} x_3  & -\varepsilon_3 + k_{13} \alpha_{13} x_1 + k_{23} \alpha_{23} x_2
        \end{matrix}\right)
    \]

    Найдём точки равновесия дифференциального уравнения, т.е. решения \( (x_1, x_2, x_3) \) системы уравнений:
    \[
        \left\{\begin{split}
            & \varepsilon_1 x_1 - \alpha_{12} x_1 x_2 - \alpha_{13} x_1 x_3 = 0, \\
            & \varepsilon_2 x_2 + k_{12} \alpha_{12} x_1 x_2 - \alpha_{23} x_2 x_3 = 0, \\
            & -\varepsilon_3 x_3 + k_{13} \alpha_{13} x_1 x_3 + k_{23} \alpha_{23} x_2 x_3 = 0. 
        \end{split}\right.
        \Rightarrow
        \left\{\begin{split}
            & x_1 (\varepsilon_1 - \alpha_{12} x_2 - \alpha_{13} x_3) = 0, \\
            & x_2 (\varepsilon_2 + k_{12} \alpha_{12} x_1 - \alpha_{23} x_3) = 0, \\
            & x_3 (-\varepsilon_3 + k_{13} \alpha_{13} x_1 + k_{23} \alpha_{23} x_2) = 0. 
        \end{split}\right.
    \]

    \begin{enumerate}
        \item Если две любых переменных равны нулю, то в оставшейся строчке остаётся уравнение \( \varepsilon_i x_i = 0, \) т.е. все переменные равны нулю. Получаем тривиальное решение \( x^{(0)} = (0,0,0) \).
        \[
            J \big|_{x^{(0)}} = \left(\begin{matrix}
                \varepsilon_1 & 0 & 0 \\
                0 & \varepsilon_2 & 0 \\
                0 & 0  & -\varepsilon_3
            \end{matrix}\right)
        \]

        Откуда получаем собственные значения матрицы: 
        \[
            \lambda_1 = \varepsilon_1 > 0, \quad \lambda_2 = \varepsilon_2 > 0, \quad \lambda_1 = -\varepsilon_3 < 0.
        \]
        Значит около начала координат в плоскостях \( x_1 = 0, x_2 = 0 \) эта точка является седлом, а в \( x_3 = 0 \) -- неустойчивым узлом.
        \item Если \( x_1 = 0; x_2, x_3 \neq 0 \):
            \[
                \left\{\begin{split}
                    & \varepsilon_2 - \alpha_{23} x_3 = 0, \\
                    & -\varepsilon_3 + k_{23} \alpha_{23} x_2 = 0. 
                \end{split}\right.
                \Rightarrow
                x^{(1)} = \left( 0,\, \frac{\varepsilon_3}{k_{23} \alpha_{23}},\, \frac{\varepsilon_2}{\alpha_{23}} \right)
            \]

            \[
                A = J \big|_{x^{(1)}} = \left(\begin{matrix}
                    \varepsilon_1 - \alpha_{12} \frac{\varepsilon_3}{k_{23} \alpha_{23}} - \alpha_{13} \frac{\varepsilon_2}{\alpha_{23}} & 0 & 0 \\[10pt]
                    k_{12} \alpha_{12} \frac{\varepsilon_3}{k_{23} \alpha_{23}} & 0 & -\alpha_{23} \frac{\varepsilon_3}{k_{23} \alpha_{23}} \\[10pt]
                    k_{13} \alpha_{13} \frac{\varepsilon_2}{\alpha_{23}} & k_{23} \alpha_{23} \frac{\varepsilon_2}{\alpha_{23}}  & 0
                \end{matrix}\right)
            \]
            \[
                \det(\lambda I - A) = \left(\lambda - \left(\varepsilon_1 - \frac{\varepsilon_3 \alpha_{12} }{k_{23} \alpha_{23}} - \frac{\varepsilon_2 \alpha_{13}}{\alpha_{23}} \right) \right)(\lambda^2 + \varepsilon_2 \varepsilon_3) = 0.
            \]
            \[
                \lambda_1 = \varepsilon_1 - \frac{\varepsilon_3 \alpha_{12} }{k_{23} \alpha_{23}} - \frac{\varepsilon_2 \alpha_{13}}{\alpha_{23}} , \quad \lambda_{2,3} = \pm i \sqrt{\varepsilon_2 \varepsilon_3}.
            \]
            Точка \( x^{(1)} \) -- неустойчивая. В плоскости \( x_1 = 0 \) точка будет являться центром (асимптотически неустойчивая точка), т.е. создавать вокруг себя замкнутые кривые.

        \item Если \( x_2 = 0; x_1, x_3 \neq 0 \):
            \[
                \left\{\begin{split}
                    & \varepsilon_1 - \alpha_{13} x_3 = 0, \\
                    & -\varepsilon_3 + k_{13} \alpha_{13} x_1 = 0. 
                \end{split}\right.
                \Rightarrow
                x^{(2)} = \left( \frac{\varepsilon_3}{k_{13} \alpha_{13}},\, 0,\, \frac{\varepsilon_1}{\alpha_{13}} \right)
            \]

            \[
                A = J \big|_{x^{(2)}} = \left(\begin{matrix}
                    0 & -\alpha_{12} \frac{\varepsilon_3}{k_{13} \alpha_{13}} & -\alpha_{13} \frac{\varepsilon_3}{k_{13} \alpha_{13}} \\[10pt]
                    0 & \varepsilon_2 + k_{12} \alpha_{12} \frac{\varepsilon_3}{k_{13} \alpha_{13}} - \alpha_{23}  \frac{\varepsilon_1}{\alpha_{13}} & 0 \\[10pt]
                    k_{13} \alpha_{13} \frac{\varepsilon_1}{\alpha_{13}} & k_{23} \alpha_{23} \frac{\varepsilon_1}{\alpha_{13}}  & 0
                \end{matrix}\right)
            \]
            \[
                \det(\lambda I - A) = \left(\lambda - \left(\varepsilon_2 + k_{12} \alpha_{12} \frac{\varepsilon_3}{k_{13} \alpha_{13}} - \alpha_{23}  \frac{\varepsilon_1}{\alpha_{13}} \right) \right)(\lambda^2 + \varepsilon_1 \varepsilon_3) = 0.
            \]
            \[
                \lambda_1 = \varepsilon_2 + k_{12} \alpha_{12} \frac{\varepsilon_3}{k_{13} \alpha_{13}} - \alpha_{23}  \frac{\varepsilon_1}{\alpha_{13}}, \quad \lambda_{2,3} = \pm i \sqrt{\varepsilon_1 \varepsilon_3}.
            \]
            Аналогично предыдущей точке, \( x^{(2)} \) -- неустойчивая и в плоскости \( x_2 = 0 \) является центром и будет создавать вокруг себя замкнутые кривые.

        \item Если \( x_3 = 0; x_1, x_2 \neq 0 \):
            \[
                \left\{\begin{split}
                    & \varepsilon_1 - \alpha_{12} x_2 = 0, \\
                    & \varepsilon_2 + k_{12} \alpha_{12} x_1 = 0. 
                \end{split}\right.
                \Rightarrow
                x^{(*)} = \left( -\frac{\varepsilon_2}{k_{12} \alpha_{12}},\, \frac{\varepsilon_1}{\alpha_{12}},\, 0 \right)
            \]
            Данная точка всегда будет находиться вне исследуемой области.

            % \[
            %     A = J \big|_{x^{(3)}} = \left(\begin{matrix}
            %         0 & -\alpha_{12} \frac{-\varepsilon_2}{k_{12} \alpha_{12}} & -\alpha_{13} \frac{-\varepsilon_2}{k_{12} \alpha_{12}} \\[10pt]
            %         k_{12} \alpha_{12} \frac{\varepsilon_1}{\alpha_{12}} & 0 & -\alpha_{23} \frac{\varepsilon_1}{\alpha_{12}} \\[10pt]
            %         0 & 0 & -\varepsilon_3 + k_{13} \alpha_{13} \frac{-\varepsilon_2}{k_{12} \alpha_{12}} + k_{23} \alpha_{23} \frac{\varepsilon_1}{\alpha_{12}}
            %     \end{matrix}\right)
            % \]
            % \[
            %     \det(\lambda I - A) = \left(\lambda - \left(-\varepsilon_3 + k_{13} \alpha_{13} \frac{-\varepsilon_2}{k_{12} \alpha_{12}} + k_{23} \alpha_{23} \frac{\varepsilon_1}{\alpha_{12}} \right) \right)(\lambda^2 - \varepsilon_1 \varepsilon_2) = 0.
            % \]
            % \[
            %     \lambda_1 = -\varepsilon_3 + k_{13} \alpha_{13} \frac{-\varepsilon_2}{k_{12} \alpha_{12}} + k_{23} \alpha_{23} \frac{\varepsilon_1}{\alpha_{12}}, \quad \lambda_{2,3} = \pm \sqrt{\varepsilon_1 \varepsilon_2}.
            % \]
            % Точка \( x^{(3)} \) -- неустойчивая, но в плоскости \( x_3 = 0 \) является седлом по некоторым двум направлениям.

        \item Если \( x_1, x_2, x_3 \neq 0 \):
            \[
                \left\{\begin{split}
                    & \varepsilon_1 - \alpha_{12} x_2 - \alpha_{13} x_3 = 0, \\
                    & \varepsilon_2 + k_{12} \alpha_{12} x_1 - \alpha_{23} x_3 = 0, \\
                    & -\varepsilon_3 + k_{13} \alpha_{13} x_1 + k_{23} \alpha_{23} x_2 = 0. 
                \end{split}\right.
            \]
            Тогда решение \( x^{(3)} \):
            \[
                \left\{\begin{split}
                    & x_1 = \frac{-\varepsilon_1 \alpha_{23} k_{23} + \varepsilon_2 k_{23} \alpha_{13} + \varepsilon_3 \alpha_{12}}{\alpha_{12} \alpha_{13} (k_{13} - k_{12} k_{23})}, \\
                    & x_2 = \frac{\varepsilon_1}{\alpha_{12}} - \frac{\alpha_{13}}{\alpha_{12}} x_3
                    = \frac{\varepsilon_1 \alpha_{23} k_{13} - \varepsilon_2 \alpha_{13} k_{13} - \varepsilon_3 \alpha_{12} k_{12}}{\alpha_{12} \alpha_{23} (k_{13} - k_{12} k_{23})}, \\ 
                    & x_3 = \frac{\varepsilon_2}{\alpha_{23}} + \frac{k_{12} \alpha_{12}}{\alpha_{23}} x_1
                    = \frac{-\varepsilon_1 \alpha_{23} k_{12} k_{23} + \varepsilon_2 \alpha_{13} k_{13} + \varepsilon_3 \alpha_{12} k_{12}}{\alpha_{13} \alpha_{23} (k_{13} - k_{12} k_{23})}.
                \end{split}\right.
            \]

            \[
                A = J \big|_{x^{(3)}} = \left(\begin{matrix}
                    0 & -\alpha_{12} x_1 & -\alpha_{13} x_1 \\
                    k_{12} \alpha_{12} x_2 & 0 & -\alpha_{23} x_2 \\
                    k_{13} \alpha_{13} x_3 & k_{23} \alpha_{23} x_3 & 0
                \end{matrix}\right)
            \]
            \[
                \det(\lambda I - A) = \lambda^3 - \lambda (k_{12} \alpha_{12}^2 x_1 x_2 + k_{13} \alpha_{13}^2 x_1 x_3 + k_{23} \alpha_{23}^2 x_2 x_3) +
            \]
            \[
                + x_1 x_2 x_3 \alpha_{12} \alpha_{13} \alpha_{23} (k_{12} k_{23} - k_{13}) = 0
            \]
            Явное решение данного уравнения будет непростым, поэтому воспользуемся критерием Рауса-Гурвица.
            \[
                b_0 = 1, \quad b_1 = 0, \quad b_2 = -(k_{12} \alpha_{12}^2 x_1 x_2 + k_{13} \alpha_{13}^2 x_1 x_3 + k_{23} \alpha_{23}^2 x_2 x_3),
            \]
            \[
                b_3 = x_1 x_2 x_3 \alpha_{12} \alpha_{13} \alpha_{23} (k_{12} k_{23} - k_{13}).
            \]
            Матрица Гурвица и главные миноры:
            \[
                \Delta = \left( \begin{matrix}
                    0 & b_3 & 0 \\
                    1 & b_2 & 0 \\
                    0   & 0 & b_3
                \end{matrix} \right)
                \Rightarrow 
                \left\{ \begin{split}
                    & \Delta_1 = 0, \\
                    & \Delta_2 = -b_3, \\
                    & \Delta_3 = b_3 \cdot \Delta_2 = -b_3^2 \leq 0.
                \end{split} \right.
            \]
            Это значит, что данная точка вне зависимости от положения будет неустойчивой.
    \end{enumerate}