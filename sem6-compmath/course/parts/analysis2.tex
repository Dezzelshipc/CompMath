\subsection{Анализ модели}
    Найдём точки равновесия дифференциального уравнения.
    \[
        \dot{x} = 0 \Rightarrow f(x) = 0
    \]

    Т.е. нужно найти решения \( (x_1, x_2, x_3) \) системы уравнений:
    \[
        \left\{\begin{split}
            & \xi x_1 - \alpha_{12} x_1 x_2 - \alpha_{13} x_1 x_3 = 0, \\
            & k_{12} \alpha_{12} x_1 x_2 - \alpha_{23} x_2 x_3 = 0, \\
            & k_{13} \alpha_{13} x_1 x_3 + k_{23} \alpha_{23} x_2 x_3 = 0. 
        \end{split}\right.
        \Rightarrow
        \left\{\begin{split}
            & x_1 (\xi - \alpha_{12} x_2 - \alpha_{13} x_3) = 0, \\
            & x_2 (k_{12} \alpha_{12} x_1 - \alpha_{23} x_3) = 0, \\
            & x_3 (k_{13} \alpha_{13} x_1 + k_{23} \alpha_{23} x_2) = 0. 
        \end{split}\right.
    \]

    \begin{enumerate}
        \item Если \( x_2 = x_3 = 0 \), то в оставшейся строчке остаётся уравнение \( \xi x_1 = 0, \) т.е. все переменные равны нулю. Получаем тривиальное решение \( x^{(0)} = (0,0,0) \).
        \item Если \( x_1 = x_2 = 0 \), то в третьей строчке получем \( x_3 \cdot 0 = 0, x_3 \in \mathbb{R} \). Получаем всю ось \( \overrightarrow{Ox_3} \).
        \item Если \( x_1 = x_3 = 0 \), то во второй строчке получем \( x_2 \cdot 0 = 0, x_2 \in \mathbb{R} \). Получаем всю ось \( \overrightarrow{Ox_2} \).
        \item Если \( x_1 = 0; x_2, x_3 \neq 0 \):
            \[
                \left\{\begin{split}
                    & \alpha_{23} x_3 = 0, \\
                    & k_{23} \alpha_{23} x_2 = 0. 
                \end{split}\right.
                \Rightarrow
                x^{(0)} = \left( 0,0, 0\right)
            \]
        \item Если \( x_2 = 0; x_1, x_3 \neq 0 \):
            \[
                \left\{\begin{split}
                    & \xi - \alpha_{13} x_3 = 0, \\
                    & k_{13} \alpha_{13} x_1 = 0. 
                \end{split}\right.
                \Rightarrow
                x^{(1)} = \left( 0, 0, \frac{\xi}{\alpha_{13}} \right) \in \overrightarrow{Ox_3}
            \]
        \item Если \( x_3 = 0; x_1, x_2 \neq 0 \):
            \[
                \left\{\begin{split}
                    & \xi - \alpha_{12} x_2 = 0, \\
                    & k_{12} \alpha_{12} x_1 = 0. 
                \end{split}\right.
                \Rightarrow
                x^{(2)} = \left( 0, \frac{\xi}{\alpha_{12}}, 0 \right)\in \overrightarrow{Ox_2}
            \]
        \item Если \( x_1, x_2, x_3 \neq 0 \):
            \[
                \left\{\begin{split}
                    & \xi - \alpha_{12} x_2 - \alpha_{13} x_3 = 0, \\
                    & k_{12} \alpha_{12} x_1 - \alpha_{23} x_3 = 0, \\
                    & k_{13} \alpha_{13} x_1 + k_{23} \alpha_{23} x_2 = 0. 
                \end{split}\right.
            \]

            Тогда решение:
            \[
                \left\{\begin{split}
                    & x_1 = \frac{-\xi \alpha_{23} k_{23}}{\alpha_{12} \alpha_{13} (k_{13} - k_{12} k_{23})}, \\
                    & x_2 = \frac{\xi k_{13}}{\alpha_{12} (k_{13} - k_{12} k_{23})}, \\ 
                    & x_3 = \frac{-\xi k_{12} k_{23}}{\alpha_{13} (k_{13} - k_{12} k_{23})}.
                \end{split}\right.
            \]
            Можем видеть, что вне зависимости от выбора параметров, хотя бы одна из координат данной точки будет отрицательная. Значит такую точку можно не исследовать далее.
    \end{enumerate}

    Получили все точки. Проведём аналогичный прошлой модели анализ устойчивости в этих точках.
    Матрица Якоби:
    \[
        J = \frac{\partial f}{\partial x} = \left(\begin{matrix}
            \xi - \alpha_{12} x_2 - \alpha_{13} x_3 & -\alpha_{12} x_1 & -\alpha_{13} x_1 \\
            k_{12} \alpha_{12} x_2 & k_{12} \alpha_{12} x_1 - \alpha_{23} x_3 & -\alpha_{23} x_2 \\
            k_{13} \alpha_{13} x_3 & k_{23} \alpha_{23} x_3  & k_{13} \alpha_{13} x_1 + k_{23} \alpha_{23} x_2
        \end{matrix}\right)
    \]

    \begin{enumerate}
        \item \(
            J \big|_{x^{(0)}} = \left(\begin{matrix}
                \xi & 0 & 0 \\
                0 & 0 & 0 \\
                0 & 0  & 0
            \end{matrix}\right)
        \)

        Откуда получаем собственные значения матрицы: 
        \[
            \lambda_1 = \xi > 0, \quad \lambda_{2,3} = 0.
        \]
        Значит около начала координат решения будут расходиться по \( x_1 \) ???

        \item \(
            A = J \big|_{\overrightarrow{Ox_3}} = \left(\begin{matrix}
                \xi - \alpha_{13} x_3 & 0 & 0 \\
                0 & - \alpha_{23} x_3 & 0 \\
                k_{13} \alpha_{13} x_3 & k_{23} \alpha_{23} x_3 & 0
            \end{matrix}\right)
        \)
        \[
            \lambda_1 = - \alpha_{23} x_3 < 0, \quad \lambda_{2} = 0, \quad \lambda_3 = \xi - \alpha_{13} x_3.
        \]
        \begin{enumerate}
            \item При \( x_3 \in \left[ 0, \frac{\xi}{\alpha_{13}} \right) \quad \lambda_3 > 0 \)
        \end{enumerate}

         Прямая  \( x^{(1)} \) -- неустойчивая. В плоскости \( x_1 = 0 \) точка будет являться центром (асимптотически неустойчивая точка), т.е. создавать вокруг себя циклы, а в некоторой близости от этой плоскости циклы будут двигаться в некотором направлении, в зависимости от констант.

        \item \(
            A = J \big|_{x^{(2)}} = \left(\begin{matrix}
                0 & -\alpha_{12} \frac{\xi_3}{k_{13} \alpha_{13}} & -\alpha_{13} \frac{\xi_3}{k_{13} \alpha_{13}} \\[10pt]
                0 & \xi_2 + k_{12} \alpha_{12} \frac{\xi_3}{k_{13} \alpha_{13}} - \alpha_{23}  \frac{\xi_1}{\alpha_{13}} & 0 \\[10pt]
                k_{13} \alpha_{13} \frac{\xi_1}{\alpha_{13}} & k_{23} \alpha_{23} \frac{\xi_1}{\alpha_{13}}  & 0
            \end{matrix}\right)
        \)
        \[
            \det(\lambda I - A) = \left(\lambda - \left(\xi_2 + k_{12} \alpha_{12} \frac{\xi_3}{k_{13} \alpha_{13}} - \alpha_{23}  \frac{\xi_1}{\alpha_{13}} \right) \right)(\lambda^2 + \xi_1 \xi_3) = 0.
        \]
        \[
            \lambda_1 = \xi_2 + k_{12} \alpha_{12} \frac{\xi_3}{k_{13} \alpha_{13}} - \alpha_{23}  \frac{\xi_1}{\alpha_{13}}, \quad \lambda_{2,3} = \pm i \sqrt{\xi_1 \xi_3}.
        \]
        Аналогично предыдущей точке, \( x^{(2)} \) -- неустойчивая и в плоскости \( x_2 = 0 \) является центром и будет создавать вокруг себя циклы.

        \item \(
            A = J \big|_{x^{(3)}} = \left(\begin{matrix}
                0 & -\alpha_{12} \frac{-\xi_2}{k_{12} \alpha_{12}} & -\alpha_{13} \frac{-\xi_2}{k_{12} \alpha_{12}} \\[10pt]
                k_{12} \alpha_{12} \frac{\xi_1}{\alpha_{12}} & 0 & -\alpha_{23} \frac{\xi_1}{\alpha_{12}} \\[10pt]
                0 & 0 & -\xi_3 + k_{13} \alpha_{13} \frac{-\xi_2}{k_{12} \alpha_{12}} + k_{23} \alpha_{23} \frac{\xi_1}{\alpha_{12}}
            \end{matrix}\right)
        \)
        \[
            \det(\lambda I - A) = \left(\lambda - \left(-\xi_3 + k_{13} \alpha_{13} \frac{-\xi_2}{k_{12} \alpha_{12}} + k_{23} \alpha_{23} \frac{\xi_1}{\alpha_{12}} \right) \right)(\lambda^2 - \xi_1 \xi_2) = 0.
        \]
        \[
            \lambda_1 = -\xi_3 + k_{13} \alpha_{13} \frac{-\xi_2}{k_{12} \alpha_{12}} + k_{23} \alpha_{23} \frac{\xi_1}{\alpha_{12}}, \quad \lambda_{2,3} = \pm \sqrt{\xi_1 \xi_2}.
        \]
        Точка \( x^{(3)} \) -- неустойчивая, но в плоскости \( x_3 = 0 \) является седлом по некоторым двум направлениям.

        \item \(
            A = J \big|_{x^{(4)}} = \left(\begin{matrix}
                0 & -\alpha_{12} x_1 & -\alpha_{13} x_1 \\
                k_{12} \alpha_{12} x_2 & 0 & -\alpha_{23} x_2 \\
                k_{13} \alpha_{13} x_3 & k_{23} \alpha_{23} x_3 & 0
            \end{matrix}\right)
        \)
        \[
            \det(\lambda I - A) = \lambda^3 - \lambda (k_{12} \alpha_{12}^2 x_1 x_2 + k_{13} \alpha_{13}^2 x_1 x_3 + k_{23} \alpha_{23}^2 x_2 x_3) +
        \]
        \[
            + x_1 x_2 x_3 \alpha_{12} \alpha_{13} \alpha_{23} (k_{12} k_{23} - k_{13}) = 0
        \]
        Явное решение данного уравнения будет непростым, поэтому воспользуемся критерием Рауса-Гурвица.
        \[
            b_0 = 1, \quad b_1 = 0, \quad b_2 = -(k_{12} \alpha_{12}^2 x_1 x_2 + k_{13} \alpha_{13}^2 x_1 x_3 + k_{23} \alpha_{23}^2 x_2 x_3),
        \]
        \[
            b_3 = x_1 x_2 x_3 \alpha_{12} \alpha_{13} \alpha_{23} (k_{12} k_{23} - k_{13}).
        \]
        Матрица Гурвица и главные миноры:
        \[
            \Delta = \left( \begin{matrix}
                0 & b_3 & 0 \\
                1 & b_2 & 0 \\
                0   & 0 & b_3
            \end{matrix} \right)
            \Rightarrow 
            \left\{ \begin{split}
                & \Delta_1 = 0, \\
                & \Delta_2 = -b_3, \\
                & \Delta_3 = b_3 \cdot \Delta_2 = -b_3^2 \leq 0.
            \end{split} \right.
        \]
    \end{enumerate}