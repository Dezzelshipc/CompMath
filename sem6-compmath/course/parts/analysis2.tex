\subsection{Анализ модели}
    Найдём точки равновесия дифференциального уравнения и исследуем их устойчивость.

    Матрица Якоби:
    \[
        J = \left(
            \begin{matrix}
                -2 \varepsilon x_1 + \delta - v_{12} x_2 - v_{13} x_3 & -v_{12}  x_1  & -v_{13} x_1 \\
                k_{12} x_2 & \left( k_{12} x_1 - m_{12} \right) - v_{23} x_3 & -v_{23} x_2 \\
                k_{13} x_3 & k_{23} x_3 & k_{13} x_1 - m_{13} + k_{23} x_2 - m_{23}
            \end{matrix}
        \right)
    \]

    Нужно найти решения \( (x_1, x_2, x_3) \) системы уравнений:
    \[
        \left\{\begin{split}
            & \left( -\varepsilon x_1 + \delta \right)x_1 - v_{12} x_1 x_2 - v_{13} x_1 x_3 = 0, \\
            & \left( k_{12} x_1 - m_{12} \right)x_2 - v_{23} x_2 x_3 = 0, \\
            & \left( k_{13} x_1 - m_{13} \right)x_3 + \left( k_{23} x_2 - m_{23} \right)x_3 = 0. 
        \end{split}\right.
    \]
    Для удобства вынесем общие множители:
    \[
        \left\{\begin{split}
            & \left( -\varepsilon x_1 + \delta - v_{12} x_2 - v_{13} x_3\right)x_1 = 0, \\
            & \left( k_{12} x_1 - m_{12} - v_{23} x_3 \right)x_2 = 0, \\
            & \left( k_{13} x_1 - m_{13} + k_{23} x_2 - m_{23} \right)x_3  = 0. 
        \end{split}\right.
    \]

    \begin{enumerate}
        \item Если \( x_2 = x_3 = 0 \), то остаётся уравнение 
        \[ ( -\varepsilon x_1 + \delta ) x_1 = 0. \] 
        Получаем тривиальное решение \( x^{(0)} = (0,0,0) \) и \( x^{(1)} = \left(\frac{\delta}{\varepsilon}, 0, 0 \right) \).
        \begin{enumerate}
            \item \( J \big|_{x^{(0)}} = \left(
                \begin{matrix}
                    \delta & 0 & 0 \\
                    0 & -m_{12}  & 0 \\
                    0 & 0 & -m_{23} (0)
                \end{matrix}
            \right) \)

            \[
                \lambda_1 = \delta> 0, ~
                \lambda_2 = -m_{12} < 0, ~
                \lambda_3 = -m_{23} < 0.
            \]

            Значит, в плоскостях \( x_2 = 0 \) и \( x_3 = 0 \) начало координат является седлом и направление \( x_1 \) -- неустойчивое. В плоскости \( x_1 = 0 \) точка является устойчивым узлом.

            \item \( J \big|_{x^{(1)}} = \left(
                \begin{matrix}
                    -\delta & -v_{12} \frac{\delta}{\varepsilon} & -v_{13} \frac{\delta}{\varepsilon} \\
                    0 & k_{12} \frac{\delta}{\varepsilon} - m_{12} & 0 \\
                    0 & 0 & k_{13} \frac{\delta}{\varepsilon} -m_{13} - m_{23}
                \end{matrix}
            \right) \)

            \[
                \lambda_1 = -\delta < 0, ~
                \lambda_2 = k_{12} \frac{\delta}{\varepsilon} - m_{12}, ~
                \lambda_3 = k_{13} \frac{\delta}{\varepsilon} -m_{13} - m_{23}. 
            \]
            Откуда имеем:
            \[
                \lambda_2 < 0 \Leftrightarrow \frac{\delta}{\varepsilon} < \frac{m_{12}}{k_{12}},
                \quad
                \lambda_3 < 0 \Leftrightarrow \frac{\delta}{\varepsilon} < \frac{m_{13} + m_{23}}{k_{13}}
            \]

            В зависимости от значений \( \lambda_2, \lambda_3 \) данная точка может быть:
            \begin{enumerate}
                \item Устойчивым узлом, если \( \lambda_2, \lambda_3 < 0 \),
                \item Устойчивым узлом в плоскости \( x_2 = 0 \) и седлом в плоскости \( x_3 = 0 \), если \( \lambda_2 > 0, \lambda_3 < 0 \),
                \item Устойчивым узлом в плоскости \( x_3 = 0 \) и седлом в плоскости \( x_2 = 0 \), если \( \lambda_2 < 0, \lambda_3 > 0 \),
                \item Седлом в плоскостях \( x_2 = 0, ~ x_3 = 0 \), если \( \lambda_2, \lambda_3 > 0 \)
            \end{enumerate}
            
            
        \end{enumerate}


        \item Если \( x_1 = x_2 = 0 \), то в третьей строчке получем
        \[ \left( -m_{13} -m_{23} \right) x_3 = 0. \]
        Поскольку \( x_3 > 0\), то данное равенство не может быть выполнено.


        \item Если \( x_1 = x_3 = 0 \), то во второй строчке получем
        \[ -m_{12} x_2 = 0. \]
        Поскольку \( x_2 > 0 \), то равенство не может быть выполнено.


        \item Если \( x_1 = 0; x_2, x_3 > 0 \):
            \[
                \left\{\begin{split}
                    & \left( -m_{12} - v_{23} x_3 \right) x_2 = 0, \\
                    & \left( -m_{13} + k_{23} x_2 - m_{23} \right) x_3 = 0. 
                \end{split}\right.
                \Rightarrow
                \left\{\begin{split}
                    & x_3 = \frac{ m_{12} }{ -v_{23} } < 0, \\
                    & x_2 = \frac{m_{13} + m_{23}}{k_{23}} . 
                \end{split}\right.
            \]
            Эта точка будет находиться вне исследуемой области.
        \item Если \( x_2 = 0; x_1, x_3 > 0 \):
            \[
                \left\{\begin{split}
                    & \left( -\varepsilon x_1 + \delta - v_{13} x_3 \right)x_1 = 0, \\
                    & \left( k_{13} x_1 -m_{13} - m_{23} \right)x_3 = 0. 
                \end{split}\right.
                \Rightarrow
                \left\{\begin{split}
                    & x_3 = \frac{ \varepsilon x_1 - \delta }{ -v_{13} }, \\
                    & x_1 = \frac{m_{13} + m_{23}}{k_{13}}. 
                \end{split}\right.
            \]
            В исследуемой области данная точка будет находиться, если \( x_1 \leq \frac{\delta}{\varepsilon} \).
            \[
                A = J \big|_{x^{(2)}} = \left(
                    \begin{matrix}
                        -\varepsilon x_1 & -v_{12}  x_1  & -v_{13} x_1 \\
                        0 & \left( k_{12} x_1 - m_{12} \right) - v_{23} x_3 & 0 \\
                        k_{13} x_3 & k_{23} x_3 & 0
                    \end{matrix}
                \right)
            \]
            
            \[
                \det(\lambda I - A) = \left( \lambda - \left( \left( k_{12} x_1 - m_{12} \right) - v_{23} x_3 \right) \right) \left( (\lambda + \varepsilon x_1)\lambda + v_{13} x_1 k_{13} x_3 \right) = 0
            \]
            \[
                \lambda_{1,3} = \frac{-\varepsilon x_1 \pm \sqrt{ (\varepsilon x_1)^2 - 4 v_{13} k_{13} x_1 x_3 } }{2}, ~ 
                \lambda_2 = \left( k_{12} x_1 - m_{12} \right) - v_{23} x_3
            \]
            % Коэффициенты характеристического полинома:
            % \[
            %     \begin{split}
            %         & b_0 = 1, \qquad b_1 = \varepsilon x_1 - k_{12} x_1 + m_{12} + v_{23} x_3, \\
            %         & b_2 = -\varepsilon k_{12} x_1^2 + \varepsilon m_{12} x_1 + \varepsilon v_{23} x_1 x_3 + k_{13} v_{13} x_1 x_3, \\
            %         & b_3 = -k_{12} k_{13} v_{13} x_1^2 x_3 + k_{13} m_{12} v_{13} x_1 x_3 + k_{13} v_{13} v_{23} x_1 x_3^2. 
            %     \end{split}
            % \]

            При разных параметрах могут быть разные комбинации устойчивости этой точки. Однако, коэффициенты в \( \lambda_{1,3} \) и \( \lambda_2 \) не пересекаются, значит существуют параметры, которые сделают эту точку устойчивым узлом или устойчивым фокусом.

        \item Если \( x_3 = 0; x_1, x_2 > 0 \):
            \[
                \left\{\begin{split}
                    & \left( -\varepsilon x_1 + \delta - v_{12} x_2 \right)x_1 = 0, \\
                    & ( k_{12} x_1 -m _{12} ) x_2 = 0. 
                \end{split}\right.
                \Rightarrow
                \left\{\begin{split}
                    & x_2 = \frac{ \varepsilon x_1 - \delta }{ -v_{12} }, \\
                    & x_1 = \frac{m_{12}}{k_{12}}. 
                \end{split}\right.
            \]
            В исследуемой области данная точка будет находиться, если \( x_1 \leq \frac{\delta}{\varepsilon}  \).
            \[
                A = J\big|_{x^{(3)}} = \left(
                    \begin{matrix}
                        -\varepsilon x_1 & -v_{12}  x_1  & -v_{13} x_1 \\
                        k_{12} x_2 & 0 & -v_{23} x_2 \\
                        0 & 0 & k_{13} x_1 - m_{13} + k_{23} x_2 - m_{23}
                    \end{matrix}
                \right)
            \]

            \[
                \det(\lambda I - A) = \left( \lambda - \left( k_{13} x_1 - m_{13} + k_{23} x_2 - m_{23} \right) \right) \left( (\lambda + \varepsilon x_1)\lambda + v_{12} x_1 k_{12} x_3 \right) = 0
            \]
            \[
                \lambda_{1,2} = \frac{-\varepsilon x_1 \pm \sqrt{ (\varepsilon x_1)^2 - 4 v_{12} k_{12} x_1 x_2 } }{2}, ~ 
                \lambda_3 = k_{13} x_1 - m_{13} + k_{23} x_2 - m_{23}
            \]

            Аналогично предыдущей точки можно подобрать параметры, чтобы точка была устойчивой.

        \item Если \( x_1, x_2, x_3 > 0 \):
            \[
                \left\{\begin{split}
                    & -\varepsilon x_1 + \delta - v_{12} x_2 - v_{13} x_3 = 0, \\
                    & k_{12} x_1 - m_{12} - v_{23} x_3 = 0, \\
                    & k_{13} x_1 - m_{13} + k_{23} x_2 - m_{23} = 0. 
                \end{split}\right.
            \]

            \[
                \left\{\begin{split}
                    & x_1 = \frac{-\delta k_{23} v_{23} - k_{23} m_{12} v_{13} + m_{13} v_{12} v_{23} + m_{23} v_{12} v_{23}}{-\varepsilon k_{23} v_{23} - k_{12} k_{23} v_{13} + k_{13} v_{12} v_{23}}, \\
                    & x_2 = \frac{\delta k_{13} v_{23} - \varepsilon m_{13} v_{23} - \varepsilon m_{23} v_{23} - k_{12} m_{13} v_{13} - k_{12} m_{23} v_{13} + k_{13} m_{12} v_{13}}{-\varepsilon k_{23} v_{23} - k_{12} k_{23} v_{13} + k_{13} v_{12} v_{23}}, \\
                    & x_3 = \frac{-\delta k_{12} k_{23} + \varepsilon k_{23} m_{12} + k_{12} m_{13} v_{12} + k_{12} m_{23} v_{12} - k_{13} m_{12} v_{12}}{-\varepsilon k_{23} v_{23} - k_{12} k_{23} v_{13} + k_{13} v_{12} v_{23}}
                \end{split}\right.
            \]

            \[
                A = J\big|_{x^{(4)}} = \left(
                    \begin{matrix}
                        -\varepsilon x_1 & -v_{12}  x_1  & -v_{13} x_1 \\
                        k_{12} x_2 & 0 & -v_{23} x_2 \\
                        k_{13} x_3 & k_{23} x_3 & 0

                    \end{matrix}
                \right)
            \]

            Для вычислительного эксперимента будем численно находить собственные значения данной точки равновесия.

            % Коэффициенты характеристического многочлена:
            % \[
            %     \begin{split}
            %         & b_0 = 1, ~ b_1 = \varepsilon x_1 \\
            %         & b_2 = k_{12} v_{12} x_1 x_2 + k_{13} v_{13} x_1 x_3 + k_{23} v_{23} x_2 x_3, \\
            %         & b_3 = \varepsilon k_{23} v_{23} x_1 x_2 x_3 + k_{12}k_{23} v_{13} x_1 x_2 x_3 - k_{13} v_{12} v_{23} x_1 x_2 x_3.
            %     \end{split}
            % \]

            % Будем проверять критерий Рауса-Гурвица для этой точки при конкретных вычислительных экспериментах.


    \end{enumerate}