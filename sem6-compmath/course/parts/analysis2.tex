\subsection{Анализ модели}
    Найдём точки равновесия дифференциального уравнения и исследуем их устойчивость.

    Матрица Якоби:
    \[
        J = \left(
            \begin{matrix}
                \varepsilon'(x_1) x_1 + \varepsilon(x_1) - V'_{12}(x_1) x_2 - V'_{13} (x_1) x_3 & -V_{12} (x_1) & -V_{13} (x_1) \\
                K'_{12} (x_1) x_2 & K_{12}(x_1) - V'_{23} (x_2) x_3 & -V_{23}(x_2) \\
                K'_{13} (x_1) x_3 & K'_{23} (x_2) x_3 & K_{23} (x_2)
            \end{matrix}
        \right)
    \]

    Нужно найти решения \( (x_1, x_2, x_3) \) системы уравнений:
    \[
        \dot{x} = 0 
        \Rightarrow 
        f(x) = 0
        \Rightarrow
        \left\{\begin{split}
            & \varepsilon(x_1)x_1 - V_{12}(x_1)x_2 - V_{13}(x_1)x_3 = 0, \\
            & K_{12}(x_1)x_2 - V_{23}(x_2)x_3 = 0, \\
            & K_{13}(x_1)x_3 + K_{23}(x_2)x_3 = 0. 
        \end{split}\right.
    \]

    \begin{enumerate}
        \item Если \( x_2 = x_3 = 0 \), то остаётся уравнение 
        \[ \varepsilon (x_1) x_1 = 0. \] 
        Получаем тривиальное решение \( x^{(0)} = (0,0,0) \) и \( x^{(1)} = (\bar{x}_1, 0, 0) \).
        \begin{enumerate}
            \item \( J \big|_{x^{(0)}} = \left(
                \begin{matrix}
                    \varepsilon(0) & 0 & 0 \\
                    0 & K_{12}(0)  & 0 \\
                    0 & 0 & K_{23} (0)
                \end{matrix}
            \right) \)

            \[
                \lambda_1 = \varepsilon(0) > 0, ~
                \lambda_2 = K_{12} (0) < 0, ~
                \lambda_3 = K_{23} (0) < 0. 
            \]

            Значит, в плоскостях \( x_2 = 0 \) и \( x_3 = 0 \) начало координат является седлом и направление \( x_1 \) -- неустойчивое. В плоскости \( x_1 = 0 \) точка является устойчивым узлом.

            \item \( J \big|_{x^{(1)}} = \left(
                \begin{matrix}
                    \varepsilon'(\bar{x}_1) \bar{x}_1 & -V_{12} (\bar{x}_1) & -V_{13} (\bar{x}_1) \\
                    0 & K_{12}(\bar{x}_1) & 0 \\
                    0 & 0 & K_{23} (0)
                \end{matrix}
            \right) \)

            \[
                \lambda_1 = \varepsilon'(\bar{x}_1) \bar{x}_1 < 0, ~
                \lambda_2 = K_{12} (\bar{x}_1), ~
                \lambda_3 = K_{23} (0) < 0. 
            \]

            В зависимости от нахождения корня функции \( K_{12} \) данная точка может быть:
            \begin{enumerate}
                \item Устойчивым узлом, если \( K_{12} (\bar{x}_1) \leq 0 \),
                \item Устойчивым узлом в плоскости \( x_2 = 0 \) и седлом в плоскости \( x_3 = 0 \), если \( K_{12} (\bar{x}_1) > 0 \),
            \end{enumerate}
            
            
        \end{enumerate}


        \item Если \( x_1 = x_2 = 0 \), то в третьей строчке получем
        \[ K_{13}(0)x_3 + K_{23}(0)x_3 = 0. \]
        Поскольку \( x_3 > 0\), то данное равенство не может быть выполнено.


        \item Если \( x_1 = x_3 = 0 \), то во второй строчке получем
        \[ K_{12}(0)x_2 = 0. \]
        Поскольку \( x_2 > 0 \), то равенство не может быть выполнено.


        \item Если \( x_1 = 0; x_2, x_3 > 0 \):
            \[
                \left\{\begin{split}
                    & K_{12}(0)x_2 - V_{23}(x_2)x_3 = 0, \\
                    & K_{13}(0)x_3 + K_{23}(x_2)x_3 = 0. 
                \end{split}\right.
                \Rightarrow
                \left\{\begin{split}
                    & x_3 = \frac{ K_{12}(0)x_2 }{ V_{23}(x_2) } < 0, \\
                    & x_2 =K^{-1}_{23} ( -K_{13}(0) ). 
                \end{split}\right.
            \]
            Поскольку функции \( K_{ij} (x_i) \) при \( x_i > 0 \) монотонно возрастающие, то можно найти обратные к ним. Однако, получили противоречие, т.к. \( K_{12}(0) < 0, V_{ij} > 0\), значит эта точка будет находиться вне исследуемой области.
        \item Если \( x_2 = 0; x_1, x_3 > 0 \):
            \[
                \left\{\begin{split}
                    & \varepsilon(x_1)x_1 - V_{13}(x_1)x_3 = 0, \\
                    & K_{13}(x_1)x_3 + K_{23}(0)x_3 = 0. 
                \end{split}\right.
                \Rightarrow
                \left\{\begin{split}
                    & x_3 = \frac{ \varepsilon(x_1)x_1 }{ V_{13}(x_1) }, \\
                    & x_1 = K^{-1}_{13} \left( - K_{23}(0) \right). 
                \end{split}\right.
            \]
            Поскольку \( K_{ij}: [0, \infty] \rightarrow [-a, \infty] \), то  \( K^{-1}_{ij}: [-a, \infty] \rightarrow [0, \infty] \), значит \( x_1 > 0 \). В исследуемой области данная точка будет находиться, если \( x_1 \leq \bar{x}_1 \).
            \[
                J \big|_{x^{(2)}} = \left(
                    \begin{matrix}
                        \varepsilon'(x_1) x_1 + \varepsilon(x_1) - V'_{13} (x_1) x_3 & -V_{12} (x_1) & -V_{13} (x_1) \\
                        0 & K_{12}(x_1) - V'_{23} (0) x_3 & 0 \\
                        K'_{13} (x_1) x_3 & K'_{23} (0) x_3 & K_{23} (0)
                    \end{matrix}
                \right)
            \]
        \item Если \( x_3 = 0; x_1, x_2 > 0 \):
            \[
                \left\{\begin{split}
                    & \varepsilon(x_1)x_1 - V_{12}(x_1)x_2 = 0, \\
                    & K_{12}(x_1)x_2 = 0. 
                \end{split}\right.
                \Rightarrow
                \left\{\begin{split}
                    & x_2 = \frac{ \varepsilon(x_1)x_1 }{ V_{12}(x_1) }, \\
                    & x_1 = K^{-1}_{12}( 0 ). 
                \end{split}\right.
            \]
            В исследуемой области данная точка будет находиться, если \( x_1 \leq \bar{x}_1 \).
            \[
                J\big|_{x^{(3)}} = \left(
                    \begin{matrix}
                        \varepsilon'(x_1) x_1 + \varepsilon(x_1) - V'_{12}(x_1) x_2 & -V_{12} (x_1) & -V_{13} (x_1) \\
                        K'_{12} (x_1) x_2 & 0 & -V_{23}(x_2) \\
                        0 & 0 & K_{23} (x_2)
                    \end{matrix}
                \right)
            \]
        \item Если \( x_1, x_2, x_3 > 0 \):
            \[
                \left\{\begin{split}
                    & \varepsilon(x_1)x_1 - V_{12}(x_1)x_2 - V_{13}(x_1)x_3 = 0, \\
                    & K_{12}(x_1)x_2 - V_{23}(x_2)x_3 = 0, \\
                    & K_{13}(x_1)x_3 + K_{23}(x_2)x_3 = 0. 
                \end{split}\right.
            \]

    \end{enumerate}