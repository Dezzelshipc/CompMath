 \subsection{Математическая модель}
    \[
        \left\{\begin{split}
            & \dot{x_1} = \varepsilon_1(x_1) - V_{12}(x_1)x_2 - V_{13}(x_1)x_3, \\
            & \dot{x_2} = \varepsilon_2(x_2) + k_{12} V_{12}(x_1)x_2 - V_{23}(x_2)x_3, \\
            & \dot{x_3} = -\varepsilon_3(x_3) + k_{13} V_{13}(x_1)x_3 + k_{23} V_{23}(x_2)x_3. 
        \end{split}\right.
    \]
    
    Имеем автономную систему \( \dot{x} = f(x) \), где \( k_{ij} > 0 \).

    Примем функции в системе за линейные функции: 
    \[ \varepsilon_i(x_j) = \varepsilon_i \cdot x_j, V_{ij}(x_k) = \alpha_{ij} \cdot x_k, \quad \varepsilon_i, \alpha_{ij} > 0 \]

    Тогда получим систему:
    \[
        \left\{\begin{split}
            & \dot{x_1} = \varepsilon_1 x_1 - \alpha_{12} x_1 x_2 - \alpha_{13} x_1 x_3, \\
            & \dot{x_2} = \varepsilon_2 x_2 + k_{12} \alpha_{12} x_1 x_2 - \alpha_{23} x_2 x_3, \\
            & \dot{x_3} = -\varepsilon_3 x_3 + k_{13} \alpha_{13} x_1 x_3 + k_{23} \alpha_{23} x_2 x_3. 
        \end{split}\right.
    \]