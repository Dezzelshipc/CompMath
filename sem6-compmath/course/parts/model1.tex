\subsection{Математическая модель}
    Классическая модель хищник-жертва для одной популяции жертвы \(x(t)\) и хищника \(y(t)\) называется такая модель\cite{svilog}:
    \begin{equation}
        \left\{\begin{split}
            & \dot{x} = \alpha(x) - V(x)y, \\
            & \dot{y} = k V(x)y -\beta(y),
        \end{split}\right. \label{lv-s}
    \end{equation}
    где коэффициенты \( \alpha, \beta \) -- функции естественного прироста жертв и естественной смертности хищников соответственно. Также их можно называть коэффициентом, обозначающим разность рождаемости и смертности в популяциях. Без взаимодеиствия жертвы беспрепятственно питаются и поэтому их рождается больше чем умирает, а у хищников -- из-за отсутствия питания -- наоборот, умирают больше, чем рождаются.

    Функция \( V(x) \) показывает количество жертв, потребляемых одним хищником за единицу времени, причём \(k\)-тая часть полученной с этой биомассой энергией расходуется хищником на воспроизводство. 
    
    При малых значениях \(x\), когда почти все жертвы становятся добычей хищника, который всегда голоден и насыщения не наступает, тогда функцию \( V(x) \) можно считать линейной функцией численности жертв: \( V = vx \). Беря функции прироста линейными, получим систему:
    \begin{equation}
        \left\{\begin{split}
            & \dot{x} = \alpha x - v x y, \\
            & \dot{y} = k v x y - \beta y,
        \end{split}\right. \label{lv-sl}
    \end{equation}
    Эта линейная система \eqref{lv-sl} совпадает с моделью В. Вольтерра, который показал, что такая система имеет неасимптотическую точку равновесия (центр) и поэтому эти популяции живут на замкнутых кривых на фазовой плоскости\cite{svilog}.

    Аналогичными предположениями приведём исследуемую схему (Рис. \ref{scheme}) к подобной системе:   
    \begin{equation}
        \left\{\begin{split}
            & \dot{x_1} = \varepsilon_1(x_1) - V_{12}(x_1)x_2 - V_{13}(x_1)x_3, \\
            & \dot{x_2} = \varepsilon_2(x_2) + k_{12} V_{12}(x_1)x_2 - V_{23}(x_2)x_3, \\
            & \dot{x_3} = -\varepsilon_3(x_3) + k_{13} V_{13}(x_1)x_3 + k_{23} V_{23}(x_2)x_3,
        \end{split}\right. \label{lv-g}
    \end{equation}
    где:
    \begin{enumerate}
        \item \( \varepsilon_i(x_i) \) -- функции прироста соответствующих популяций без взаимодействия с другими. Предполагаем, что жертва и первый хищник \(( x_1, x_2) \) имеют положительный прирост, а второй хищник \(( x_3 )\) -- отрицательный.
        \item \( V_{ij} (x_i) \) -- функции, показывающие какое количество из популяции \(x_i\) поглощается одним хищником популяции \( x_j \) за единицу времени. \(k_{ij}\) -- соответствующие части полученной при поглощении энергии, которые идут на воспроизводство популяции \(x_j\).
    \end{enumerate}
 
    Имеем автономную систему \( \dot{x} = f(x) \), где \( k_{ij} > 0 \). Аналогично примем функции в системе \eqref{lv-g} за линейные функции: 
    \[ \varepsilon_i(x_j) = \varepsilon_i \cdot x_j, ~ V_{ij}(x_k) = \alpha_{ij} \cdot x_k, \quad \varepsilon_i, \alpha_{ij} > 0 \]

    Тогда получим систему:
    \begin{equation}
        \left\{\begin{split}
            & \dot{x_1} = \varepsilon_1 x_1 - \alpha_{12} x_1 x_2 - \alpha_{13} x_1 x_3, \\
            & \dot{x_2} = \varepsilon_2 x_2 + k_{12} \alpha_{12} x_1 x_2 - \alpha_{23} x_2 x_3, \\
            & \dot{x_3} = -\varepsilon_3 x_3 + k_{13} \alpha_{13} x_1 x_3 + k_{23} \alpha_{23} x_2 x_3. 
        \end{split}\right. \label{lv-gl}
    \end{equation}

    Проанализируем, какое поведение популяций будет в этой модели.