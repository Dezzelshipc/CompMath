\section{Заключение}
    В курсовой работе была рассмотрена схема взаимодействия трех популяций животных: жертвы и двух хищников. Для неё были построены две модели конкуренции, основанные на модели Лотки-Вольтерры и модели Колмогорова. В ходе анализа этих моделей были найдены точки равновесия системы и исследована их устойчивость. На основе анализа были сделаны выводы о примерном поведении популяций, что было показано в ходе вычислительных экспериментов.

    На основании этой работы можно сделать вывод, что первая построенная модель (аналог модели Лотки-Вольтерры) описывала такое поведение популяций, при котором, в общем случае, одна из них вымирала, оставляя две другие конкурировать с периодическими циклами их объёмов биомассы. Вторая модель (аналог модели Колмогорова) при ненулевых популяциях описывает поведение, при котором все три популяции со временем придут к балансу и их биомасса перестанет изменяться.